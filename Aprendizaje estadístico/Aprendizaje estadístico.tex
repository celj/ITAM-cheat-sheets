% !TEX program = lualatex

\documentclass[8pt,a4paper]{extarticle}
\usepackage[utf8]{inputenc}
\usepackage[spanish]{babel}
\usepackage[landscape, margin=1cm, bmargin=0.5cm, includefoot, footskip=0.5cm]{geometry}
\usepackage[textsize=tiny]{todonotes}
\usepackage{enumitem}
\usepackage{mdframed}
\usepackage{mathtools}
\usepackage{amsthm}
\usepackage{amssymb}
\usepackage{multicol,multirow}
\usepackage{subfiles}
\usepackage{tabularx}
\usepackage{bm}
\usepackage{xcolor}
\usepackage{graphicx}
\usepackage{accents}
\usepackage{pgfplots}
\usepackage{fancyhdr}
\usepackage[hidelinks]{hyperref}
\usepackage{nicefrac}
\usepackage{fontspec}
\usepackage{listings}

\newcommand{\cs}{Formulario}
\newcommand{\csof}{Formulario de }
\newcommand{\csAuthorName}{Carlos Lezama}
\newcommand{\csClass}{ }
\newcommand{\csClassCode}{ }
\newcommand{\csKeywords}{ }
\newcommand{\csTerm}{ }
\newcommand{\csSchool}{ITAM}

\pagestyle{fancy}
\renewcommand{\headrulewidth}{0pt}
\rhead{} 
\lhead{} 
\chead{} 
\cfoot{\csClass\ $\cdot$ \cs}
\lfoot{\csAuthorName}
\rfoot{Página \thepage}

\graphicspath{{./figures/}}

\setmonofont{JetBrainsMono Nerd Font Mono}[
    Contextuals = Alternate,
    Ligatures = TeX,
]

\lstset{
    basicstyle = \ttfamily,
    columns = flexible,
}

\usetikzlibrary{decorations.markings}
\pgfplotsset{compat=1.11}

\newmdtheoremenv [
	topline    = false,
	bottomline = false,
	leftline   = true,
	rightline  = false,
	linewidth  = 2.5pt,
	linecolor  = red!50
]{boxdef}{Definición}[section]

\mdtheorem [
	topline    = false,
	bottomline = false,
	leftline   = true,
	rightline  = false,
	linewidth  = 2.5pt,
	linecolor  = blue!40
]{boxtheo}{Teorema}[section]

\mdtheorem [
	topline    = false,
	bottomline = false,
	leftline   = true,
	rightline  = false,
	linewidth  = 2.5pt,
	linecolor  = black!20
]{boxprop}{Proposición}[section]

\mdtheorem [
	topline    = false,
	bottomline = false,
	leftline   = true,
	rightline  = false,
	linewidth  = 2.5pt,
	linecolor  = blue!40
]{boxlemma}{Lema}[section]

\mdtheorem [
	topline    = false,
	bottomline = false,
	leftline   = true,
	rightline  = false,
	linewidth  = 2.5pt,
	linecolor  = blue!40
]{boxcor}{Corolario}[section]

\mdtheorem [
	topline    = false,
	bottomline = false,
	leftline   = true,
	rightline  = false,
	linewidth  = 2.5pt,
	linecolor  = black!20
]{boxrmk}{Observación}[section]

\newlist{numberlist}{enumerate}{1}
\setlist[numberlist, 1]{label={\arabic*.}, itemsep=0em, leftmargin=*,labelindent=0.5em}

\newlist{eqlist}{enumerate}{1}
\setlist[eqlist, 1]{label={\normalfont (\roman*)},itemsep=-0.2em, leftmargin=*,labelindent=-0.5em}

\newlist{bulletlist}{itemize}{1}
\setlist[bulletlist, 1]{itemsep=0em, leftmargin=0.5em, label={·}}

\setlength{\parindent}{0em}

\newenvironment{Figure}
  {\par\medskip\noindent\minipage{\linewidth}}
  {\endminipage\par\medskip}

\newcommand\tab[1][0.5em]{\hspace*{#1}}

\newcommand{\sectionbreak}{\vfill\ \columnbreak}

\usepackage{array}
\newcolumntype{P}[1]{>{\centering\arraybackslash}p{#1}}
\newcolumntype{M}[1]{>{\centering\arraybackslash}m{#1}}


% Economics
\newcommand{\E}{\resizebox{0.2cm}{!}{$\varepsilon$}}
\newcommand{\EE}{\mathcal{E}}
\newcommand{\I}{\mathcal{I}}
\newcommand{\LL}{\mathcal{L}}
\newcommand{\F}{\mathcal{F}}
\newcommand{\MRS}{\text{\normalfont MRS}}

% Statistics

% Mathematics
\DeclareMathOperator*{\argmin}{\arg \min}
\DeclareMathOperator*{\argmax}{\arg \max}
\newcommand{\deq}{\stackrel{\text{\normalfont def}}{=}}
\newcommand{\ie}{\text{\normalfont i.e.}}
\newcommand{\sgn}{\text{\normalfont sgn}}



% Class info
\renewcommand{\csClass}{Aprendizaje estadístico}
\renewcommand{\csClassCode}{EST - 25134}
\renewcommand{\csTerm}{Primavera 2021}
\renewcommand{\csKeywords}{ }

% PDF Metadata
\hypersetup{
    pdftitle={\csof \csClass},      
    pdfsubject={\csClass},      
    pdfauthor={\csAuthorName},  
    pdfkeywords={}              
}

% Begin document
\begin{document}

\begin{titlepage}
    \begin{center}
	\vspace*{1cm}
	\Huge
        \textbf{\csClass}
	\vspace{0.5cm} \\
	\Large
        \cs\ $\cdot$ \csTerm
        \vfill
        \csAuthorName
	\vspace{0.8cm}
        \csClassCode\\
        \csSchool     
    \end{center}
\end{titlepage}

\begin{multicols}{3}
\setcounter{page}{1}

\section*{Introducción}

\subsection*{?`Cuándo necesitamos aprendizaje de máquina?}

Dos aspectos de un problema dado pueden requerir el uso de programas que aprendan y mejoren sobre la base de su ``experiencia'': la \textbf{complejidad} del problema y la necesidad de \textbf{adaptabilidad}.

\subsubsection*{Complejidad}

\begin{bulletlist}
\item Tareas realizadas por animales o humanos.
\item Tareas más allá de las capacidades humanas.
\end{bulletlist}

\subsubsection*{Adaptabilidad}

Una característica limitante de las herramientas programadas es su rigidez: una vez que el programa se ha escrito e instalado, permanece sin cambios. Sin embargo, muchas tareas cambian con el tiempo, o de un usuario a otro. Las herramientas de aprendizaje de máquina (programas cuyo comportamiento se adapta a sus datos de entrada) ofrecen una solución a estos problemas; estos son, por naturaleza, adaptables a los cambios en el entorno con el que interactúan.

\subsection*{Tipos de aprendizaje}

\begin{bulletlist}
\item Supervisado o no supervisado.
\item Por refuerzo.
\item Agentes pasivos o activos.
\item Maestro.
\item Protocolo por bloques o continuo.
\end{bulletlist}

\newpage

\section{Modelo formal de aprendizaje}

\subsection{Marco formal de aprendizaje estadístico}

\subsubsection*{Entradas ($\left\{ \mathcal{X}, \mathcal{Y}, S \right\} $)}

\begin{bulletlist}
\item \textbf{Conjunto de dominio} ($\mathcal{X}$): $\mathcal{X} \subseteq \mathbb{R}^d$ tal que $d < \infty$.
\item \textbf{Conjunto de etiquetas} ($\mathcal{Y}$): en el caso de etiquetado binario, podemos definir $\mathcal{Y} = \left\{ 0,1 \right\} $ ó $\mathcal{Y} = \left\{ -1,+1 \right\} $.
\item \textbf{Conjunto de entrenamiento} ($S$): Una sucesión $\displaystyle S = \left\{ \left( x_i, y_i \right)  \right\}_{i = 1}^m $ tal que $m < \infty$ y $\left( x_i, y_i \right) \in \mathcal{X} \times \mathcal{Y}$.
\end{bulletlist}

\subsubsection*{Reglas de predicción ($h$)}

$h : \mathcal{X} \to \mathcal{Y}$ ; también llamado \emph{predictor}, \emph{hipótesis} o \emph{clasificador}.

\subsubsection*{Algoritmo de aprendizaje ($A$)}

Denotamos $A(S)$ a la hipótesis que el algoritmo de aprendizaje $A$ genera al observar el conjunto de entrenamiento $S$. Asimismo, asumimos que $\mathcal{X}$ tiene una medida de probabilidad desconocida $\mathcal{D}$ y que existe una función desconocida $f$ que etiqueta los datos de manera correcta, es decir:
\[
	\exists f:\mathcal{X}\to \mathcal{Y}\quad \text{\normalfont tal que} \quad f(x_i) = y_i,\; \forall i
.\] 

\subsubsection*{Métricas de éxito}

\begin{boxdef}[Error de un clasificador]
	Dado un subconjunto de dominio $A \subseteq \mathcal{X}$ y su probabilidad de observarlo $\mathcal{D}(A)$. En muchos casos, nos referimos a $A$ como un evento y lo expresamos usando una función $\pi : \mathcal{X} \to \{0,1\}$ tal que $A = \{x \in \mathcal{X} : \pi(x) = 1\} $. Así pues, definimos el \textbf{error del clasificador} $h : \mathcal{X} \to \mathcal{Y}$ como sigue:
	\[
		L_{\mathcal{D}, f} (h) \deq \underset{x \sim \mathcal{D}}{\mathbb{P}} \left[ h(x) \neq f(x) \right] \deq \mathcal{D}\left( \{x : h(x) \neq f(x)\}  \right) 
	.\]
	También se le conoce como \textbf{error de generalización}, \textbf{riesgo} o \textbf{error verdadero} de $h$.
\end{boxdef}

\sectionbreak

\subsection{Minimización de riesgo empírico}



\vfill\eject
\columnbreak
\end{multicols}
\end{document}
