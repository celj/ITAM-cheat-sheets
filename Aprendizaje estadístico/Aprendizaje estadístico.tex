% !TEX program = xelatex

\documentclass[8pt,a4paper]{extarticle}
\usepackage[utf8]{inputenc}
\usepackage[spanish]{babel}
\usepackage[landscape, margin=1cm, bmargin=0.5cm, includefoot, footskip=0.5cm]{geometry}
\usepackage[textsize=tiny]{todonotes}
\usepackage{enumitem}
\usepackage{mdframed}
\usepackage{mathtools}
\usepackage{amsthm}
\usepackage{amssymb}
\usepackage{multicol,multirow}
\usepackage{subfiles}
\usepackage{tabularx}
\usepackage{bm}
\usepackage{xcolor}
\usepackage{graphicx}
\usepackage{accents}
\usepackage{pgfplots}
\usepackage{fancyhdr}
\usepackage[hidelinks]{hyperref}
\usepackage{nicefrac}

\newcommand{\cs}{Formulario}
\newcommand{\csof}{Formulario de }
\newcommand{\csAuthorName}{Carlos Lezama}
\newcommand{\csClass}{ }
\newcommand{\csClassCode}{ }
\newcommand{\csKeywords}{ }
\newcommand{\csTerm}{ }
\newcommand{\csSchool}{ITAM}

\pagestyle{fancy}
\renewcommand{\headrulewidth}{0pt}
\rhead{} 
\lhead{} 
\chead{} 
\cfoot{\csClass\ $\cdot$ \cs}
\lfoot{\csAuthorName}
\rfoot{Página \thepage}

\graphicspath{{./figures/}}

\usetikzlibrary{decorations.markings}
\pgfplotsset{compat=1.11}

\newmdtheoremenv [
	topline    = false,
	bottomline = false,
	leftline   = true,
	rightline  = false,
	linewidth  = 2.5pt,
	linecolor  = red!50
]{boxdef}{Definición}[section]

\mdtheorem [
	topline    = false,
	bottomline = false,
	leftline   = true,
	rightline  = false,
	linewidth  = 2.5pt,
	linecolor  = blue!40
]{boxtheo}{Teorema}[section]

\mdtheorem [
	topline    = false,
	bottomline = false,
	leftline   = true,
	rightline  = false,
	linewidth  = 2.5pt,
	linecolor  = black!20
]{boxprop}{Proposición}[section]

\mdtheorem [
	topline    = false,
	bottomline = false,
	leftline   = true,
	rightline  = false,
	linewidth  = 2.5pt,
	linecolor  = blue!40
]{boxlemma}{Lema}[section]

\mdtheorem [
	topline    = false,
	bottomline = false,
	leftline   = true,
	rightline  = false,
	linewidth  = 2.5pt,
	linecolor  = blue!40
]{boxcor}{Corolario}[section]

\newlist{numberlist}{enumerate}{1}
\setlist[numberlist, 1]{label={\arabic*.}, itemsep=0em, leftmargin=*,labelindent=0.5em}

\newlist{eqlist}{enumerate}{1}
\setlist[eqlist, 1]{label={\normalfont (\roman*)},itemsep=-0.2em, leftmargin=*,labelindent=-0.5em}

\newlist{bulletlist}{itemize}{1}
\setlist[bulletlist, 1]{itemsep=0em, leftmargin=0.5em, label={·}}

\setlength{\parindent}{0em}

\newenvironment{Figure}
  {\par\medskip\noindent\minipage{\linewidth}}
  {\endminipage\par\medskip}

\newcommand\tab[1][0.5em]{\hspace*{#1}}

\newcommand{\sectionbreak}{\vfill\ \columnbreak}

\usepackage{array}
\newcolumntype{P}[1]{>{\centering\arraybackslash}p{#1}}
\newcolumntype{M}[1]{>{\centering\arraybackslash}m{#1}}


% Economics
\newcommand{\E}{\resizebox{0.2cm}{!}{$\varepsilon$}}
\newcommand{\EE}{\mathcal{E}}
\newcommand{\I}{\mathcal{I}}
\newcommand{\LL}{\mathcal{L}}
\newcommand{\F}{\mathcal{F}}
\newcommand{\MRS}{\text{\normalfont MRS}}

% Statistics
\newcommand{\bias}{\text{\normalfont Bias}}
\newcommand{\corr}{\text{\normalfont Corr}}
\newcommand{\cov}{\text{\normalfont Cov}}
\newcommand{\var}{\text{\normalfont Var}}

% Mathematics
\newcommand{\ie}{\text{\normalfont i.e.}}
\newcommand{\sgn}{\text{\normalfont sgn}}



% Class info
\renewcommand{\csClass}{Aprendizaje estadístico}
\renewcommand{\csClassCode}{EST - 25134}
\renewcommand{\csTerm}{Primavera 2021}
\renewcommand{\csKeywords}{ }

% PDF Metadata
\hypersetup{
    pdftitle={\csof \csClass},      
    pdfsubject={\csClass},      
    pdfauthor={\csAuthorName},  
    pdfkeywords={}              
}

% Begin document
\begin{document}

\begin{titlepage}
	\begin{center}
		\vspace*{1cm}
		\Huge
		\textbf{\csClass}
		\vspace{0.5cm} \\
		\Large
		\cs\ $\cdot$ \csTerm
		\vfill
		\csAuthorName
		\vspace{0.8cm}
		\csClassCode\\
		\csSchool
	\end{center}
\end{titlepage}

\begin{multicols}{3}
	\setcounter{page}{1}

	\section*{Introducción}

	\subsection*{?`Cuándo necesitamos aprendizaje de máquina?}

	Dos aspectos de un problema dado pueden requerir el uso de programas que aprendan y mejoren sobre la base de su ``experiencia'': la \textbf{complejidad} del problema y la necesidad de \textbf{adaptabilidad}.

	\subsubsection*{Complejidad}

	\begin{bulletlist}
		\item Tareas realizadas por animales o humanos.
		\item Tareas más allá de las capacidades humanas.
	\end{bulletlist}

	\subsubsection*{Adaptabilidad}

	Una característica limitante de las herramientas programadas es su rigidez: una vez que el programa se ha escrito e instalado, permanece sin cambios. Sin embargo, muchas tareas cambian con el tiempo, o de un usuario a otro. Las herramientas de aprendizaje de máquina (programas cuyo comportamiento se adapta a sus datos de entrada) ofrecen una solución a estos problemas; estos son, por naturaleza, adaptables a los cambios en el entorno con el que interactúan.

	\subsection*{Tipos de aprendizaje}

	\begin{bulletlist}
		\item Supervisado o no supervisado.
		\item Por refuerzo.
		\item Agentes pasivos o activos.
		\item Maestro.
		\item Protocolo por bloques o continuo.
	\end{bulletlist}

	\newpage

	\section{Marco formal de aprendizaje}

	\subsubsection*{Conjunto de dominio ($\mathcal{X}$)}

	$$\mathcal{X} \subseteq \mathbb{R}^d\quad \text{\normalfont tal que} \quad d < \infty.$$

	\subsubsection*{Conjunto de etiquetas ($\mathcal{Y}$)}

	En el caso de etiquetado binario, \par $$\mathcal{Y} = \left\{ 0,1 \right\}\quad \text{\normalfont ó}\quad \mathcal{Y} = \left\{ -1,+1 \right\} .$$

	\subsubsection*{Conjunto de entrenamiento ($S$)}

	Una sucesión $\displaystyle S = \left\{ \left( x_i, y_i \right)  \right\}_{i = 1}^m $ tal que $m < \infty$ y $\left( x_i, y_i \right) \in \mathcal{X} \times \mathcal{Y}$.

	\subsubsection*{Reglas de predicción ($h$)}

	$$h : \mathcal{X} \to \mathcal{Y},$$ también llamado \emph{predictor}, \emph{hipótesis} o \emph{clasificador}.

	\subsubsection*{Algoritmo de aprendizaje ($A$)}

	Denotamos $A(S)$ a la hipótesis que el algoritmo de aprendizaje $A$ genera al observar el conjunto de entrenamiento $S$. Asimismo, asumimos que $\mathcal{X}$ tiene una medida de probabilidad desconocida $\mathcal{D}$ y que existe una función desconocida $f$ que etiqueta los datos de manera correcta, es decir:
	\[
		\exists f:\mathcal{X}\to \mathcal{Y}\quad \text{\normalfont tal que} \quad f(x_i) = y_i,\; \forall i
		.\]

	\subsubsection*{Métricas de éxito}

	\begin{boxdef}[Error de un clasificador]
		Dado un subconjunto de dominio $A \subseteq \mathcal{X}$ y su probabilidad de observarlo $\mathcal{D}(A)$. En muchos casos, nos referimos a $A$ como un evento y lo expresamos usando una función $\pi : \mathcal{X} \to \{0,1\}$ tal que $A = \{x \in \mathcal{X} : \pi(x) = 1\} $. Así pues, definimos el \textbf{error del clasificador} $h : \mathcal{X} \to \mathcal{Y}$ como sigue:
		\[
			L_{\mathcal{D}, f} (h) \deq \underset{x \sim \mathcal{D}}{\mathbb{P}} \left[ h(x) \neq f(x) \right] \deq \mathcal{D}\left( \{x : h(x) \neq f(x)\}  \right)
			.\]
		También se le conoce como \textbf{error de generalización} o \textbf{riesgo} de $h$.
	\end{boxdef}

	\subsection{Minimización de riesgo empírico}

	El objetivo de nuestro algoritmo es encontrar $h_S : \mathcal{X} \to \mathcal{Y}$ que minimice el error con respecto a las desconocidas $\mathcal{D}$ y $f$.

	\begin{boxdef}[Error de entrenamiento]
		El error en el que incurre el clasificador sobre el conjunto de entrenamiento está dado por:
		\[
			L_S (h) \deq \frac{\left| i \in [m] : h(x_i) \neq y_i \right| }{m}
			,\]
		donde
		\[
			[m] = \{1,\ldots, m\}
			.\]
		A este \textbf{error de entrenamiento} también se le conoce como \textbf{error empírico} o \textbf{riesgo empírico}.
	\end{boxdef}

	\begin{boxdef}[Minimización de riesgo empírico]
		Al predictor $h$ que minimiza $L_S (h)$ se le conoce como \textbf{minimizador de riesgo empírico} o \textbf{ERM}, por sus siglas en inglés.
	\end{boxdef}

	\begin{boxdef}[Sobreajuste]
		Cuando un sistema se sobreentrena, o se entrena con datos extraños, el algoritmo de aprendizaje puede quedar ajustado a unas características muy específicas de los datos de entrenamiento que no tienen relación causal con la función objetivo. A este fenómeno le decimos \textbf{sobreajuste}.
	\end{boxdef}

	\subsection{Minimización de riesgo empírico con sesgo inductivo}

	\begin{boxdef}[Familia de hipótesis]
		Sea $\Omega$ el espacio de todos los clasificadores, definimos una \textbf{familia} (o clase) \textbf{de hipótesis} como el conjunto $\mathcal{H} \subset \Omega$ que el agente elige de antemano como un espacio de búsqueda restringido.
	\end{boxdef}

	Dada una clase $\mathcal{H}$ y un conjunto de entrenamiento $S$, el agente  $\text{\normalfont ERM}_{\mathcal{H}}$ usa la regla de \emph{minimización de riesgo empírico} para escoger un predictor $h \in \mathcal{H}$ con el menor error posible sobre $S$.

	\[
		\ie\quad\text{\normalfont ERM}_{\mathcal{H}} (S) \in \argmin_{h\in\mathcal{H}} L_S (h)
		.\]

	Así pues, a tales restricciones se les conoce como \textbf{sesgo inductivo}.

	\subsection{Familia de hipótesis finita}

	\begin{boxrmk}[]
		Denotamos por $h_S$ al resultado generado de aplicar  $\text{\normalfont ERM}_{\mathcal{H}}$ al conjunto de entrenamiento $S$.
		\[
			\ie\quad h_S \in \argmin_{h \in \mathcal{H}} L_S (h)
			.\]
	\end{boxrmk}

	\begin{boxdef}[Hipótesis de realizabilidad]
		Existe $h^* \in \mathcal{H}$ tal que $L_{\mathcal{D}, f} (h^*) = 0$. Asimismo, podemos decir que $L_S (h^*) = 0$ con probabilidad 1 sobre  $S$ cuando las muestras de $S$ se distribuyen  $\mathcal{D}$ y se etiquetan $f$.
	\end{boxdef}

	\begin{boxdef}[Hipótesis de independencia y distribución idéntica]
		Asumimos que cada $x_i \in S$ se distribuye  $\mathcal{D}$, es decir:
		\[
			S \sim \mathcal{D}^m
			,\]
		donde $m = \left| S \right| $.
	\end{boxdef}

	\begin{boxrmk}[Parámetro de confianza]
		Denotamos por $\delta$ la probabilidad de obtener una muestra poco representativa de  $\mathcal{D}$. Así pues, podemos ver a $(1 - \delta)$ como nuestro \textbf{parámetro de confianza}.
	\end{boxrmk}

	\begin{boxrmk}[Parámetro de precisión]
		Denotamos por $\varepsilon$ a la probabilidad de encontrar errores de etiquetado. Es decir, interpretamos  $L_{\mathcal{D}, f} (h_S) > \varepsilon$ como un fracaso para el agente y $L_{\mathcal{D}, f} (h_S) \le \varepsilon$ como un clasificador aproximadamente correcto.
	\end{boxrmk}

	\begin{boxlemma}[]
		Para dos conjuntos cualesquiera $A$ y $B$, y una distribución $\mathcal{D}$, tenemos:
		\[
			\mathcal{D}\left( A \cup B \right) \le \mathcal{D}(A) + \mathcal{D}(B)
			.\]
	\end{boxlemma}

	\begin{boxcor}[]
		Sean $\mathcal{H}$ una familia de hipótesis finita, $\delta \in (0,1)$,  $\varepsilon > 0$ y $m$ un entero, se cumple:
		\[
			m \ge \frac{\log \left( \nicefrac{\left| \mathcal{H} \right| }{\delta} \right)  }{\varepsilon}
			.\]
		Entonces, para toda función de etiquetado $f$ y toda distribución  $\mathcal{D}$, para las cuales se cumple la \emph{hipótesis de realizabilidad}, tenemos --- con probabilidad de al menos $(1 - \delta)$ sobre nuestra muestra independiente e idénticamente distribuida $S$ de tamaño $m$ --- que toda solución del minimizador de riesgo empírico $h_S$ satisface:
		\[
			L_{\mathcal{D}, f} (h_S) \le \varepsilon
			.\]
	\end{boxcor}

	\newpage

	\section{Un modelo formal de aprendizaje}

	\subsection{Aprendizaje PAC}

	\begin{boxdef}[Aprendizaje PAC]
		Una familia de hipótesis $\mathcal{H}$ es \textbf{PAC\footnote{Probablemente aproximadamente correcto, del inglés: \emph{Probably Approximately Correct}.} aprendible} si existe una función $m_{\mathcal{H}} : (0,1)^2 \to \mathbb{N}$ y un algoritmo de aprendizaje con la siguiente propiedad: \textcolor{dred}{para toda $\varepsilon, \delta \in (0,1)$, toda distribución $\mathcal{D}$ sobre $ \mathcal{X}$ y toda función de etiquetado $f : \mathcal{X} \to \left\{ 0,1 \right\} $, si se satisface la hipótesis de realizabilidad con respecto a $\mathcal{H}$, $\mathcal{D}$ y $f$, al ejecutar el algoritmo de aprendizaje con  $m \ge m_{\mathcal{H}} (\varepsilon, \delta)$ elementos independientes e idénticamente distribuidos por $\mathcal{D}$ y etiquetados por $f$, el algoritmo genera una hipótesis  $h$ tal que, con probabilidad de al menos  $(1 - \delta)$ sobre la elección de elementos,  $L_{\mathcal{D}, f} (h) \le \varepsilon$.}
	\end{boxdef}

	\begin{boxdef}[Complejidad muestral]
		La \textbf{complejidad muestral} de un algoritmo de aprendizaje representa la cantidad de muestras de entrenamiento que necesita para aprender con \emph{éxito} una función objetivo.
	\end{boxdef}

	\begin{boxcor}[]
		Toda familia finita de hipótesis es PAC aprendible con complejidad muestral:
		\[
			m_{\mathcal{H}}(\varepsilon, \delta) \le \left\lceil \frac{\log \left( \nicefrac{\left| \mathcal{H} \right| }{\delta} \right) }{\varepsilon} \right\rceil
			.\]
	\end{boxcor}

	\subsection{Aprendizaje PAC agnóstico}

	\begin{boxdef}[Error verdadero de un clasificador]
		Para alguna distribución de probabilidad $\mathcal{D}$ sobre $\mathcal{X} \times \mathcal{Y}$ se puede medir la probabilidad de que $h$ cometa un error cuando los puntos etiquetados se extraen aleatoriamente con respecto a $\mathcal{D}$. Así pues, definimos el \textbf{error} (o riesgo) \textbf{verdadero del clasificador} $h$, como sigue:
		\[
			L_{\mathcal{D}}(h) \deq \underset{(x,y)\sim \mathcal{D}}{\mathbb{P}} [h(x) \neq y] \deq \mathcal{D}\left( \{(x, y) : h(x) \neq y\}  \right)
			.\]
	\end{boxdef}

	\emph{Nota:} la definición de \textbf{error de entrenamiento} no se modifica.

	\begin{boxdef}[Clasificador bayesiano óptimo]
		Dada una distribución de probabilidad $\mathcal{D}$ sobre $\mathcal{X} \times \{0, 1\} $, el mejor clasificador es:
		\[
			f_{\mathcal{D}}(x) = \begin{cases}
				1, & \text{\normalfont si } \mathbb{P}[y = 1  \mid x] \ge \nicefrac{1}{2} \\
				0, & \text{\normalfont en otros casos}
			\end{cases}
			.\]
	\end{boxdef}

	\begin{boxrmk}[]
		El \emph{clasificador bayesiano} es óptimo porque cualquier otro clasificador $g : \mathcal{X} \to \{0,1\} $ tiene un error mayor.
		\[
			\ie \quad L_{\mathcal{D}}(f_{\mathcal{D}}) \le L_{\mathcal{D}}(g), \; \forall g
			.\]
	\end{boxrmk}

	\begin{boxdef}[Aprendizaje PAC agnóstico]
		Una familia de hipótesis $\mathcal{H}$ es \textbf{PAC aprendible} en un sentido \textbf{agnóstico} si existe una función $m_{\mathcal{H}} : (0,1)^2 \to \mathbb{N}$ y un algoritmo de aprendizaje con la siguiente propiedad: \textcolor{dred}{para toda $\varepsilon, \delta \in (0,1)$ y toda distribución $\mathcal{D}$ sobre $\mathcal{X} \times \mathcal{Y}$, al ejecutar el algoritmo de aprendizaje con $m \ge m_{\mathcal{H}} (\varepsilon, \delta)$ elementos independientes e idénticamente distribuidos por $\mathcal{D}$, el algoritmo genera una hipótesis $h$ tal que, con probabilidad de al menos  $(1 - \delta)$ sobre la elección de $m$ elementos de entrenamiento, $\displaystyle L_{\mathcal{D}} (h) \le \min_{h' \in \mathcal{H}} L_{\mathcal{D}} (h') + \varepsilon$.}
	\end{boxdef}

	\subsection{El alcance de los problemas de aprendizaje modelados}

	\begin{boxrmk}[Error cuadrático medio]
		Podemos evaluar la calidad de un clasificador $h : \mathcal{X} \to \mathcal{Y}$ por el \textbf{error cuadrático medio} entre el etiquetado correcto y los valores predichos.
		\[
			\ie \quad L_{\mathcal{D}}(h) \deq \underset{(x,y)\sim \mathcal{D}}{\mathbb{E}} \left( h(x) - y \right)^2
			.\]
	\end{boxrmk}

	\begin{boxdef}[Función de pérdida generalizada]
		Dado cualquier conjunto $\mathcal{H}$ y algún dominio $Z$. $ \ell$ es cualquier función de $\mathcal{H} \times Z$ a los reales no negativos; $\ie\; \ell: \mathcal{H} \times Z \to \mathbb{R}_{+}$. A dichas funciones las llamamos \textbf{funciones de pérdida}. \par
		Nótese que, para los problemas de predicción, $\mathcal{H}$ es nuestra familia de hipótesis y $Z = \mathcal{X} \times \mathcal{Y}$.
	\end{boxdef}

	\begin{boxdef}[Función de riesgo]
		Definimos la \textbf{función de riesgo} como la pérdida esperada de un clasificador $h \in \mathcal{H}$ con respecto a una distribución de probabilidad $\mathcal{D}$ sobre $Z$. Es decir:
		\[
			L_{\mathcal{D}} (h) \deq \underset{z\sim \mathcal{D}}{\mathbb{E}} \left[ \ell (h,z) \right]
			.\]
	\end{boxdef}

	\begin{boxdef}[Riesgo empírico]
		Definimos el \textbf{riesgo empírico} como la pérdida esperada sobre una muestra $S = \left( z_1, \ldots, z_m \right) \in Z^m$. Es decir:
		\[
			L_S (h) \deq \frac{1}{m} \sum_{i = 1}^{m} \ell (h, z_i)
			.\]
	\end{boxdef}

	\begin{boxprop}[]
		El \emph{riesgo empírico} $L_S (h)$ es un estimador insesgado.
	\end{boxprop}

	\begin{boxdef}[Pérdida 0-1]
		\[
			\ell_{0-1} (h, (x,y)) \deq \begin{cases}
				0, & \text{\normalfont si } h(x) = y    \\
				1, & \text{\normalfont si } h(x) \neq y
			\end{cases}
			.\]
	\end{boxdef}

	\begin{boxdef}[Pérdida cuadrática]
		\[
			\ell_{\text{\normalfont sq}} (h, (x,y)) \deq \left( h(x) - y \right)^2
			.\]
	\end{boxdef}

	\begin{boxdef}[Aprendizaje PAC agnóstico para funciones de pérdida generalizada]
		Una familia de hipótesis $\mathcal{H}$ es \textbf{PAC aprendible} en un sentido \textbf{agnóstico} con respecto a $Z$ y $\ell: \mathcal{H}\times Z \to \mathbb{R}_{+}$ (una función de pérdida) si existe una función $m_{\mathcal{H}} : (0,1)^2 \to \mathbb{N}$ y un algoritmo de aprendizaje con la siguiente propiedad: \textcolor{dred}{para toda $\varepsilon, \delta \in (0,1)$ y toda distribución $\mathcal{D}$ sobre $Z$, al ejecutar el algoritmo de aprendizaje con $m \ge m_{\mathcal{H}} (\varepsilon, \delta)$ elementos independientes e idénticamente distribuidos por $\mathcal{D}$, el algoritmo genera una hipótesis $h \in \mathcal{H}$ tal que, con probabilidad de al menos  $(1 - \delta)$ sobre la elección de $m$ elementos de entrenamiento, $L_{\mathcal{D}} (h) \le \min_{h' \in \mathcal{H}} L_{\mathcal{D}} (h') + \varepsilon$, donde $L_{\mathcal{D}} (h) = \underset{z\sim \mathcal{D}}{\mathbb{E}} [\ell (h,z)]$}
	\end{boxdef}

	\begin{boxrmk}[]
		En aprendizaje PAC agnóstico para funciones de pérdida generalizada, necesitamos que la función $ \ell (h,\cdot)$ sea \emph{medible}. Es decir, al asumir un $\sigma$-álgebra de subconjuntos de  $Z$ sobre el cuál la probabilidad  $\mathcal{D}$ está definida, y que la preimagen de cada segmento inicial en $\mathbb{R}_{+}$ está en este $\sigma$-álgebra. \textcolor{dgray}{En el caso específico de clasificación binaria con pérdida 0-1, la  $\sigma$-álgebra está sobre  $ \mathcal{X} \times \{0,1\} $ y nuestra suposición en $ \ell$ es equivalente a asumir que sobre toda $h$, el conjunto  $ \{(x, h(x)) : x \in \mathcal{X}\} $ está en el $\sigma$-álgebra.}
	\end{boxrmk}

	\begin{boxrmk}[Aprendizaje adecuado o independiente de la representación]
		En algunos casos, $\mathcal{H}$ es un subconjunto de algún conjunto $\mathcal{H}'$ y la función de pérdida puede extenderse a ser una función de $\mathcal{H}' \times Z$ a los reales. En este caso, podemos permitir que el algoritmo genere una hipótesis $h' \in \mathcal{H}'$ siempre y cuando cumpla con $L_{\mathcal{D}} (h') \le \min_{h \in \mathcal{H}} L_{\mathcal{D}} (h) + \varepsilon$. Al permitir que el algoritmo genere una hipótesis de $\mathcal{H}'$, se dice que dicho aprendizaje es de \emph{representación independiente} o \emph{inadecuado}; mientras que, en aprendizaje \emph{adecuado}, el algoritmo debe producir una hipótesis de $\mathcal{H}$.
	\end{boxrmk}

	\newpage

	\section{Convergencia uniforme}

	\begin{boxdef}[Muestra $\varepsilon$-representativa]
		Un conjunto de entrenamiento $S$ es \textbf{$\bm{\varepsilon}$-representativo} respecto al dominio $Z$, la familia de hipótesis $\mathcal{H}$, la función de pérdida $ \ell$ y la distribución $\mathcal{D}$ si:
		\[
			\forall h \in \mathcal{H}, \quad \left| L_S (h) - L_{\mathcal{D}} (h) \right| \le \varepsilon
			.\]
	\end{boxdef}

	\begin{boxlemma}[]
		Al asumir $S$ un conjunto de entrenamiento $\nicefrac{\varepsilon}{2}$-representativo, cualquier $\displaystyle h_{S} \in \argmin_{h \in \mathcal{H}} L_{S} (h)$ satisface:
		\[
			L_{\mathcal{D}}(h_{S}) \le \min_{h \in \mathcal{H}} L_{\mathcal{D}} (h) + \varepsilon
			.\]
	\end{boxlemma}

	\begin{boxdef}[Convergencia uniforme]
		Decimos que una familia de hipótesis $\mathcal{H}$ tiene la propiedad de \textbf{convergencia uniforme} con respecto al dominio $Z$ y la función de pérdida  $ \ell$ si existe una función $m_{\mathcal{H}}^{\text{\normalfont UC}} : (0,1)^2 \to \mathbb{N}$ tal que, para toda $\varepsilon, \delta \in (0,1)$ y toda distribución de probabilidad  $\mathcal{D}$ sobre $Z$, si $S$ es una muestra de  $m \ge m_{\mathcal{H}}^{\text{\normalfont UC}}$ elementos independientes e idénticamente distribuidos por $\mathcal{D}$, entonces, con probabilidad de al menos $(1 - \delta)$,  $S$ es $\varepsilon$-representativa.
	\end{boxdef}

	\begin{boxcor}[]
		Si una familia de hipótesis $\mathcal{H}$ tiene convergencia uniforme con una función $m_{\mathcal{H}}^{\text{\normalfont UC}}$, entonces la familia es PAC aprendible en sentido agnóstico con una complejidad muestral $m_{\mathcal{H}} (\varepsilon, \delta) \le m_{\mathcal{H}}^{\text{\normalfont UC}} (\nicefrac{\varepsilon}{2}, \delta)$.
	\end{boxcor}

	\begin{boxlemma}[Desigualdad de Hoeffding]
		Sea $\theta_1, \ldots, \theta_m$ una sucesión de variables aleatorias independientes e idénticamente distribuidas. Al asumir que, para toda $i$, $\mathbb{E}[\theta_i] = \mu$ y $\mathbb{P}[a \le \theta_i \le b] = 1$; para toda $\varepsilon > 0$:
		\[
			\mathbb{P} \left[ \left| \frac{1}{m} \sum_{i = 1}^{m} \theta_i - \mu \right| > \varepsilon \right] \le 2 \exp \left( \frac{-2m \varepsilon^2}{(b - a)^2} \right)
			.\]
	\end{boxlemma}

	\begin{boxcor}[]
		Sean $\mathcal{H}$ un clase de hipótesis finita, $Z$ el dominio y  $ \ell : \mathcal{H} \times Z \to [0,1]$ una función de pérdida. Entonces, $\mathcal{H}$ posee la propiedad de convergencia uniforme con complejidad muestral:
		\[
			m_{\mathcal{H}}^{\text{\normalfont UC}} (\varepsilon, \delta) \le \left\lceil \frac{\log(\nicefrac{2\left| \mathcal{H} \right| }{\delta})}{2\varepsilon^2} \right\rceil
			.\]
		Asimismo, la clase es PAC aprendible en un sentido agnóstico al utilizar el algoritmo ERM con complejidad muestral:
		\[
			m_{\mathcal{H}} (\varepsilon, \delta) \le m_{\mathcal{H}}^{\text{\normalfont UC}} (\nicefrac{\varepsilon}{2}, \delta) \le \left\lceil \frac{2\log(\nicefrac{2\left| \mathcal{H} \right| }{\delta})}{\varepsilon^2} \right\rceil
			.\]
	\end{boxcor}

	\newpage

	\section{Predictores lineales}

	\begin{boxdef}[Clase de transformaciones afines]
		Definimos la \textbf{clase de transformaciones afines} como:
		\[
			L_{d} = \{ h_{\mathbf{w}, b} : \mathbf{w} \in \mathbb{R}^d, b \in \mathbb{R} \}
			,\]
		donde
		\[
			h_{\mathbf{w}, b} (\mathbf{x}) = \left< \mathbf{w}, \mathbf{x} \right> + b = \left( \sum_{i = 1}^{d} w_i x_i \right) + b
			.\]
		Asimismo, es conveniente usar la siguiente notación:
		\[
			L_{d} = \{ \mathbf{x} \mapsto \left< \mathbf{w}, \mathbf{x} \right> + b : \mathbf{w} \in \mathbb{R}^d, b \in \mathbb{R} \}
			.\]
	\end{boxdef}

	\begin{boxrmk}[]
		En algunos casos, resulta conveniente incorporar el \emph{sesgo} $b$ en  $ \mathbf{w}$ como una coordenada y agregar una coordenada extra con valor de 1 a todas las $ \mathbf{x}  \in \mathcal{X}$. Es decir, sean $ \mathbf{w}' = (b, w_1, w_2, \ldots, w_d) \in \mathbb{R}^{d + 1}$ y $ \mathbf{x}' = (1, x_1, x_2, \ldots, x_d) \in \mathbb{R}^{d + 1}$. Por lo tanto:
		\[
			h_{\mathbf{w}, b} (\mathbf{x}) = \left< \mathbf{w}, \mathbf{x} \right> + b = \left< \mathbf{w}', \mathbf{x}' \right>
			.\]
	\end{boxrmk}

	\subsection{Semiespacios}

	\begin{boxdef}[Clase de semiespacios]
		La \textbf{clase de semiespacios}, diseñada para problemas de clasificación binaria con $\mathcal{X} = \mathbb{R}^d$ y $\mathcal{Y} = \{-1, +1\} $, la definimos como:
		\[
			HS_d = \sgn \circ L_{d} = \{\mathbf{x} \mapsto \sgn(h_{\mathbf{w}, b} (\mathbf{x})) : h_{\mathbf{w}, b} \in L_d\}
			.\]
	\end{boxdef}

	Podemos considerar las siguientes tres situaciones:

	\begin{numberlist}
		\item Casos separables, al asumir que se cumple la \emph{hipótesis de realizabilidad}.
		\item Casos no separables en un sentido agnóstico.
		\item[\textcolor{dred}{3.}] Cuando una función lineal no es suficiente y necesitamos una transformación no lineal.
	\end{numberlist}

	\subsubsection{Programación lineal}

	\begin{boxdef}[Programas lineales]
		Los \textbf{programas lineales} son problemas que pueden expresarse como la maximización de una función lineal sujeta a restricciones lineales. Es decir:
		\begin{align*}
			\max_{\mathbf{w} \in \mathbb{R}^d} \quad & \left<\mathbf{u}, \mathbf{w} \right> \\
			\text{\normalfont sujeto a} \quad        & A\mathbf{w} \ge \mathbf{v}
			,\end{align*}
		donde $ \mathbf{w}$ es el vector de variables que deseamos determinar, $A$ es una matriz  $m \times d$ y $ \mathbf{v} \in \mathbb{R}^m$, $ \mathbf{u} \in \mathbb{R}^d$ son vectores.
	\end{boxdef}

	\begin{boxprop}[]
		Sea $\displaystyle S = \left\{ \left( \mathbf{x}_i, y_i \right)  \right\}_{i = 1}^m $ un conjunto de entrenamiento de tamaño $m$. Al asumir casos separables, existe  $ \mathbf{w} \in \mathbb{R}^d$ tal que:
		\[
			y_i \left< \mathbf{w}, \mathbf{x}_i \right> \ge 1, \quad \forall i = 1, \ldots, m
			,\]
		o bien,
		\[
			A\mathbf{w} \ge \mathbf{v}
			,\]
		donde $A \in \mathbb{R}^{m \times d}$, $A_{i,j} = y_i\, x_{i,j}$ y $ \mathbf{v} = (1, \ldots, 1) \in \mathbb{R}^m$; y $ \mathbf{w}$ es un predictor ERM.
	\end{boxprop}

	\subsubsection{Algoritmo de perceptrones}

	\begin{algorithm}[H]
		\caption{Perceptrón por bloques}
		\KwIn{Un conjunto de entrenamiento $S = \{(\mathbf{x}_i, y_i)\}_{i=1}^m $}
		$ \mathbf{w}^{(1)} = (0, \ldots, 0)$\\
		\For{$t = 1, 2, \ldots$}{
			\eIf{$\exists i$ tal que $y_i\left<\mathbf{w}^{(t)}, \mathbf{x}_i \right> \le 0$}{
				$ \mathbf{w}^{(t + 1)} = \mathbf{w}^{(t)} + y_i \mathbf{x}_i$
			}{
				\KwOut{$ \mathbf{w}^{(t)}$}
			}
		}
	\end{algorithm}

	\begin{boxtheo}[]
		Al asumir $\left\{\left(\mathbf{x}_i, y_i \right)\right\}_{i=1}^m$ separables, y sean:
		$$B = \min \left\{ \| \mathbf{w} \| : y_i \langle \mathbf{w}, \mathbf{x}_i \rangle \ge 1 \right\},$$
		y
		$$\displaystyle R = \max_{i} \left\{ \| \mathbf{x}_i \| \right\}.$$
		Entonces, el algortimo de perceptrones se detiene a lo más $\left( RB \right)^2 $ iteraciones después y devuelve $\mathbf{w}^{(t)}$ tal que $y_i \langle \mathbf{w}^{(t)}, \mathbf{x}_i \rangle > 0$.
	\end{boxtheo}

	\vfill\eject
	\columnbreak
\end{multicols}
\end{document}
