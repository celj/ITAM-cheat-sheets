% !TEX program = lualatex

\documentclass[8pt,a4paper]{extarticle}
\usepackage[utf8]{inputenc}
\usepackage[spanish]{babel}
\usepackage[landscape, margin=1cm, bmargin=0.5cm, includefoot, footskip=0.5cm]{geometry}
\usepackage[textsize=tiny]{todonotes}
\usepackage{enumitem}
\usepackage{mdframed}
\usepackage{mathtools}
\usepackage{amsthm}
\usepackage{amssymb}
\usepackage{multicol,multirow}
\usepackage{subfiles}
\usepackage{tabularx}
\usepackage{bm}
\usepackage{xcolor}
\usepackage{graphicx}
\usepackage{accents}
\usepackage{pgfplots}
\usepackage{fancyhdr}
\usepackage[hidelinks]{hyperref}
\usepackage{nicefrac}
\usepackage{fontspec}
\usepackage{listings}

\newcommand{\cs}{Formulario}
\newcommand{\csof}{Formulario de }
\newcommand{\csAuthorName}{Carlos Lezama}
\newcommand{\csClass}{ }
\newcommand{\csClassCode}{ }
\newcommand{\csKeywords}{ }
\newcommand{\csTerm}{ }
\newcommand{\csSchool}{ITAM}

\pagestyle{fancy}
\renewcommand{\headrulewidth}{0pt}
\rhead{} 
\lhead{} 
\chead{} 
\cfoot{\csClass\ $\cdot$ \cs}
\lfoot{\csAuthorName}
\rfoot{Página \thepage}

\graphicspath{{./figures/}}

\setmonofont{JetBrainsMono Nerd Font Mono}[
    Contextuals = Alternate,
    Ligatures = TeX,
]

\lstset{
    basicstyle = \ttfamily,
    columns = flexible,
}

\usetikzlibrary{decorations.markings}
\pgfplotsset{compat=1.11}

\newmdtheoremenv [
	topline    = false,
	bottomline = false,
	leftline   = true,
	rightline  = false,
	linewidth  = 2.5pt,
	linecolor  = red!50
]{boxdef}{Definición}[section]

\mdtheorem [
	topline    = false,
	bottomline = false,
	leftline   = true,
	rightline  = false,
	linewidth  = 2.5pt,
	linecolor  = blue!40
]{boxtheo}{Teorema}[section]

\mdtheorem [
	topline    = false,
	bottomline = false,
	leftline   = true,
	rightline  = false,
	linewidth  = 2.5pt,
	linecolor  = black!20
]{boxprop}{Proposición}[section]

\mdtheorem [
	topline    = false,
	bottomline = false,
	leftline   = true,
	rightline  = false,
	linewidth  = 2.5pt,
	linecolor  = blue!40
]{boxlemma}{Lema}[section]

\mdtheorem [
	topline    = false,
	bottomline = false,
	leftline   = true,
	rightline  = false,
	linewidth  = 2.5pt,
	linecolor  = blue!40
]{boxcor}{Corolario}[section]

\mdtheorem [
	topline    = false,
	bottomline = false,
	leftline   = true,
	rightline  = false,
	linewidth  = 2.5pt,
	linecolor  = black!20
]{boxrmk}{Observación}[section]

\newlist{numberlist}{enumerate}{1}
\setlist[numberlist, 1]{label={\arabic*.}, itemsep=0em, leftmargin=*,labelindent=0.5em}

\newlist{eqlist}{enumerate}{1}
\setlist[eqlist, 1]{label={\normalfont (\roman*)},itemsep=-0.2em, leftmargin=*,labelindent=-0.5em}

\newlist{bulletlist}{itemize}{1}
\setlist[bulletlist, 1]{itemsep=0em, leftmargin=0.5em, label={·}}

\setlength{\parindent}{0em}

\newenvironment{Figure}
  {\par\medskip\noindent\minipage{\linewidth}}
  {\endminipage\par\medskip}

\newcommand\tab[1][0.5em]{\hspace*{#1}}

\newcommand{\sectionbreak}{\vfill\ \columnbreak}

\usepackage{array}
\newcolumntype{P}[1]{>{\centering\arraybackslash}p{#1}}
\newcolumntype{M}[1]{>{\centering\arraybackslash}m{#1}}


% Economics
\newcommand{\E}{\resizebox{0.2cm}{!}{$\varepsilon$}}
\newcommand{\EE}{\mathcal{E}}
\newcommand{\I}{\mathcal{I}}
\newcommand{\LL}{\mathcal{L}}
\newcommand{\F}{\mathcal{F}}
\newcommand{\MRS}{\text{\normalfont MRS}}

% Statistics

% Mathematics
\DeclareMathOperator*{\argmin}{\arg \min}
\DeclareMathOperator*{\argmax}{\arg \max}
\newcommand{\deq}{\stackrel{\text{\normalfont def}}{=}}
\newcommand{\ie}{\text{\normalfont i.e.}}
\newcommand{\sgn}{\text{\normalfont sgn}}



\hfuzz=12pt
\hbadness=10000

% Class info
\renewcommand{\csClass}{Cálculo numérico 1}
\renewcommand{\csClassCode}{MAT $\cdot$ 14400}
\renewcommand{\csTerm}{Primavera 2021}
\renewcommand{\csKeywords}{ }

% PDF Metadata
\hypersetup{
    pdftitle={\csof \csClass},
    pdfsubject={\csClass},
    pdfauthor={\csAuthorName},
    pdfkeywords={}
}

% Begin document
\begin{document}

\begin{titlepage}
	\begin{center}
		\vspace*{1cm}
		\Huge
		\textbf{\csClass}
		\vspace{0.5cm} \\
		\Large
		\cs\ $\cdot$ \csTerm
		\vfill
		\csAuthorName\\
		\vspace{0.8cm}
		\csClassCode\\
		\csSchool
	\end{center}
\end{titlepage}

\begin{multicols}{3}
	\setcounter{page}{1}

	\section*{Fundamentos}

	\subsection{Aritmética de punto flotante}

	Un número de punto flotante consiste en tres partes:

	\begin{numberlist}
		\item el \textbf{signo} ($+$ o $-$);
		\item la \textbf{mantisa}, que contiene la cadena de bits significativos;
		\item el \textbf{exponente}.
	\end{numberlist}

	\begin{center}
		\begin{tabular}{| c || c | c | c | c | c || c |}
			\hhline{-||-----||-}
			$\pm$ & $d_1$ & $d_2$ & $\cdots$ & $d_{m-1}$ & $d_m$ & $s$ \\
			\hhline{-||-----||-}
		\end{tabular}
	\end{center}

	\[
		\ie \quad \pm 0.d_1 d_2 \ldots d_{m-1} d_m \times 10^s, \quad m \in \mathbb{N},\ s \in \mathbb{Z}
		.\]

	\subsubsection*{Formatos y precisión}

	\begin{center}
		\begin{tabular}{| c || c | c | c || c |}
			\hhline{-||---||-}
			Precisión          & Signo & Mantisa ($m$) & Exponente ($s$) & Bits totales \\
			\hhline{=::===::=}
			\emph{half}        & 1     & 10            & 5               & 16           \\
			\emph{single}      & 1     & 23            & 8               & 32           \\
			\emph{double}      & 1     & 52            & 11              & 64           \\
			\emph{long double} & 1     & 64            & 15              & 80           \\
			\emph{quad}        & 1     & 112           & 15              & 128          \\
			\hhline{-||---||-}
		\end{tabular}
	\end{center}

	\begin{boxprop}[]
		El número positivo más grande está dado por:
		\[
			X_{\max} = \left( 1 - 10^{-m} \right) 10^{E_{\max}}
			,\]
		donde $E_{\max}$ es el exponente entero positivo más grande.
	\end{boxprop}

	\begin{boxprop}[]
		El número positivo más pequeño está dado por:
		\[
			X_{\min} = 10^{E_{\min} - 1}
			,\]
		donde $E_{\min}$ es el exponente entero negativo más pequeño.
	\end{boxprop}

	\sectionbreak

	\begin{boxdef}[Método de corte]
		Sea algún número real \[x = (0.d_1 d_2 \ldots d_m d_{m + 1} \ldots) \times 10^s,\]
		\[
			fl(x)= (0.d_1 d_2 \ldots d_m) \times 10^s
			.\]
	\end{boxdef}

	\begin{boxdef}[Método de redondeo]
		Sea algún número real
		\[x = (0.d_1 d_2 \ldots d_m d_{m + 1} \ldots) \times 10^s,\]
		\[
			fl(x) = \begin{cases}
				(0.d_1 d_2 \ldots d_m) \times 10^s,                              & \text{\normalfont si } d_{m+1} < 5   \\
				(0.d_1 d_2 \ldots d_m) \times 10^s + (0.0 \ldots 01)\times 10^s, & \text{\normalfont si } d_{m+1} \ge 5
			\end{cases}
			.\]
	\end{boxdef}

	\begin{boxtheo}[]
		Al usar el método de corte, para toda $x \in \left[ X_{\min}, X_{\max} \right]$ se cumple:
		\[
			\frac{\left| x - fl(x) \right| }{ \left| x \right|  } \le 10^{1 - m}
			.\]
	\end{boxtheo}

	\begin{boxdef}[Operación suma]
		Definimos el algoritmo de la \textbf{suma} como sigue:
		\\
		\\
		\begin{algorithm}[H]
			\KwIn{$x, y$}
			\KwOut{$\tilde{z}$}
			$\tilde{x} \longleftarrow fl(x)$\;
			$\tilde{y} \longleftarrow fl(y)$\;
			$z \longleftarrow \tilde{x} + \tilde{y}$\;
			$\tilde{z} \longleftarrow fl(z)$
		\end{algorithm}
	\end{boxdef}

	\begin{boxdef}[Épsilon de máquina]
		La \emph{precisión numérica} es el primer número positivo $\varepsilon$ tal que:
		\[
			1 \oplus \varepsilon > 1
			.\]
	\end{boxdef}

	\newpage

	\section{Localización de raíces y extremos locales}

	\begin{algorithm}[H]
		\caption{Método de bisección}
		\KwIn{$a$, $b$, $k_{\max}$, TOL}
		\KwOut{$c$, $k$}
		$k \longleftarrow 0$\;
		\While{$\left| b - a \right| > \text{\normalfont TOL}$ {\bfseries\upshape and} $k < k_{\max}$}{
			$c \longleftarrow (a + b) / 2$\;
			\If{
				$f(c) = 0$
			}{
				\bfseries\upshape stop
			}
			\eIf{
				$f(a) f(b) < 0$
			}{
				$b \longleftarrow c$\;
			}{
				$a \longleftarrow c$\;
			}
			$k \longleftarrow k + 1$\;
		}
	\end{algorithm}

	\vspace{1cm}

	\begin{algorithm}[H]
		\caption{Iteración de punto fijo}
		\KwIn{$x_0$, $k_{\max}$, TOL}
		\KwOut{$k, x_{k}$}
		$k \longleftarrow 0$\;
		\While{$\left| g(x_{k}) \right| > \text{\normalfont TOL}$ {\bfseries\upshape and} $k < k_{\max}$}{
			$x_{k + 1} \longleftarrow g(x_k)$\;
			$k \longleftarrow k + 1$\;
		}
	\end{algorithm}

	\sectionbreak

	\begin{algorithm}[H]
		\caption{Método de Newton}
		\KwIn{$x_0$, $k_{\max}$, TOL}
		\KwOut{$k, x_{k}$}
		$k \longleftarrow 0$\;
		\While{$\left| f(x_{k}) \right| > \text{\normalfont TOL}$ {\bfseries\upshape and} $k < k_{\max}$}{
			$\displaystyle x_{k + 1} \longleftarrow x_k - \frac{f(x_k)}{f'(x_k)}$\;
			$k \longleftarrow k + 1$\;
		}
	\end{algorithm}

	\vspace{1cm}

	\begin{algorithm}[H]
		\caption{Método de la secante}
		\KwIn{$x_0$, $x_1$, $k_{\max}$, TOL}
		\KwOut{$k, x_{k}$}
		$k \longleftarrow 1$\;
		\While{$\left| f(x_{k}) \right| > \text{\normalfont TOL}$ {\bfseries\upshape and} $k < k_{\max}$}{
			$\displaystyle x_{k + 1} \longleftarrow x_k - \frac{f(x_k) (x_k - x_{k - 1})}{f(x_k) - f(x_{k - 1})}$\;
			$k \longleftarrow k + 1$\;
		}
	\end{algorithm}

	\sectionbreak

	\begin{algorithm}[H]
		\caption{Regula Falsi}
		\KwIn{$a$, $b$, $k_{\max}$, TOL}
		\KwOut{$c$, $k$}
		$k \longleftarrow 0$\;
		\While{$\left| b - a \right| > \text{\normalfont TOL}$ {\bfseries\upshape and} $k < k_{\max}$}{
			$\displaystyle c \longleftarrow \frac{bf(a) - af(b)}{f(a) - f(b)}$\;
			\If{
				$f(c) = 0$
			}{
				\bfseries\upshape stop
			}
			\eIf{
				$f(a) f(b) < 0$
			}{
				$b \longleftarrow c$\;
			}{
				$a \longleftarrow c$\;
			}
			$k \longleftarrow k + 1$\;
		}
	\end{algorithm}

	\vfill\eject
	\columnbreak
\end{multicols}
\end{document}
