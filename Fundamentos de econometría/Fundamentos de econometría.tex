% !TEX program = lualatex

\documentclass[8pt,a4paper]{extarticle}
\usepackage[utf8]{inputenc}
\usepackage[spanish]{babel}
\usepackage[landscape, margin=1cm, bmargin=0.5cm, includefoot, footskip=0.5cm]{geometry}
\usepackage[textsize=tiny]{todonotes}
\usepackage{enumitem}
\usepackage{mdframed}
\usepackage{mathtools}
\usepackage{amsthm}
\usepackage{amssymb}
\usepackage{multicol,multirow}
\usepackage{subfiles}
\usepackage{tabularx}
\usepackage{bm}
\usepackage{xcolor}
\usepackage{graphicx}
\usepackage{accents}
\usepackage{pgfplots}
\usepackage{fancyhdr}
\usepackage[hidelinks]{hyperref}
\usepackage{nicefrac}
\usepackage{fontspec}
\usepackage{listings}

\newcommand{\cs}{Formulario}
\newcommand{\csof}{Formulario de }
\newcommand{\csAuthorName}{Carlos Lezama}
\newcommand{\csClass}{ }
\newcommand{\csClassCode}{ }
\newcommand{\csKeywords}{ }
\newcommand{\csTerm}{ }
\newcommand{\csSchool}{ITAM}

\pagestyle{fancy}
\renewcommand{\headrulewidth}{0pt}
\rhead{} 
\lhead{} 
\chead{} 
\cfoot{\csClass\ $\cdot$ \cs}
\lfoot{\csAuthorName}
\rfoot{Página \thepage}

\graphicspath{{./figures/}}

\setmonofont{JetBrainsMono Nerd Font Mono}[
    Contextuals = Alternate,
    Ligatures = TeX,
]

\lstset{
    basicstyle = \ttfamily,
    columns = flexible,
}

\usetikzlibrary{decorations.markings}
\pgfplotsset{compat=1.11}

\newmdtheoremenv [
	topline    = false,
	bottomline = false,
	leftline   = true,
	rightline  = false,
	linewidth  = 2.5pt,
	linecolor  = red!50
]{boxdef}{Definición}[section]

\mdtheorem [
	topline    = false,
	bottomline = false,
	leftline   = true,
	rightline  = false,
	linewidth  = 2.5pt,
	linecolor  = blue!40
]{boxtheo}{Teorema}[section]

\mdtheorem [
	topline    = false,
	bottomline = false,
	leftline   = true,
	rightline  = false,
	linewidth  = 2.5pt,
	linecolor  = black!20
]{boxprop}{Proposición}[section]

\mdtheorem [
	topline    = false,
	bottomline = false,
	leftline   = true,
	rightline  = false,
	linewidth  = 2.5pt,
	linecolor  = blue!40
]{boxlemma}{Lema}[section]

\mdtheorem [
	topline    = false,
	bottomline = false,
	leftline   = true,
	rightline  = false,
	linewidth  = 2.5pt,
	linecolor  = blue!40
]{boxcor}{Corolario}[section]

\mdtheorem [
	topline    = false,
	bottomline = false,
	leftline   = true,
	rightline  = false,
	linewidth  = 2.5pt,
	linecolor  = black!20
]{boxrmk}{Observación}[section]

\newlist{numberlist}{enumerate}{1}
\setlist[numberlist, 1]{label={\arabic*.}, itemsep=0em, leftmargin=*,labelindent=0.5em}

\newlist{eqlist}{enumerate}{1}
\setlist[eqlist, 1]{label={\normalfont (\roman*)},itemsep=-0.2em, leftmargin=*,labelindent=-0.5em}

\newlist{bulletlist}{itemize}{1}
\setlist[bulletlist, 1]{itemsep=0em, leftmargin=0.5em, label={·}}

\setlength{\parindent}{0em}

\newenvironment{Figure}
  {\par\medskip\noindent\minipage{\linewidth}}
  {\endminipage\par\medskip}

\newcommand\tab[1][0.5em]{\hspace*{#1}}

\newcommand{\sectionbreak}{\vfill\ \columnbreak}

\usepackage{array}
\newcolumntype{P}[1]{>{\centering\arraybackslash}p{#1}}
\newcolumntype{M}[1]{>{\centering\arraybackslash}m{#1}}


% Economics
\newcommand{\E}{\resizebox{0.2cm}{!}{$\varepsilon$}}
\newcommand{\EE}{\mathcal{E}}
\newcommand{\I}{\mathcal{I}}
\newcommand{\LL}{\mathcal{L}}
\newcommand{\F}{\mathcal{F}}
\newcommand{\MRS}{\text{\normalfont MRS}}

% Statistics

% Mathematics
\DeclareMathOperator*{\argmin}{\arg \min}
\DeclareMathOperator*{\argmax}{\arg \max}
\newcommand{\deq}{\stackrel{\text{\normalfont def}}{=}}
\newcommand{\ie}{\text{\normalfont i.e.}}
\newcommand{\sgn}{\text{\normalfont sgn}}



% Class info
\renewcommand{\csClass}{Fundamentos de econometría}
\renewcommand{\csClassCode}{EST - 21104}
\renewcommand{\csTerm}{Primavera 2021}
\renewcommand{\csKeywords}{ }

% PDF Metadata
\hypersetup{
    pdftitle={\csof \csClass},      
    pdfsubject={\csClass},      
    pdfauthor={\csAuthorName},  
    pdfkeywords={}              
}

% Begin document
\begin{document}

\begin{titlepage}
    \begin{center}
	\vspace*{1cm}
	\Huge
        \textbf{\csClass}
	\vspace{0.5cm} \\
	\Large
        \cs\ $\cdot$ \csTerm
        \vfill
        \csAuthorName
	\vspace{0.8cm}
        \csClassCode\\
        \csSchool     
    \end{center}
\end{titlepage}

\begin{multicols}{3}
\setcounter{page}{1}

\section{Fundamentos estadísticos de econometría}

\begin{boxdef}[Error cuadrático medio]
	Definimos el \textbf{error cuadrático medio} como sigue:
	\[
		\text{\normalfont ECM} (\hat{\theta}) = \mathbb{E}_{\theta} \left[ \left( \hat{\theta} - \theta \right)^2  \right]  = \text{\normalfont Var} (\hat{\theta}) + \text{\normalfont Bias}^{2} (\hat{\theta}, \theta)
	.\] 
	Este mide la diferencia en media cuadrada entre nuestros valores estimados y el real. Asimismo, podemos definir el \textbf{error cuadrático medio} de una variable $X$ respecto a alguna constante $c$ como sigue:
	\[
		\mathbb{E} \left[ (X - c)^2 \right] = \sigma^2_{\mathbf{x}} + (c - \mu_\mathbf{x})^2
	.\] 
\end{boxdef} 

\begin{boxtheo}[Error cuadrático mínimo]
	El valor que minimiza $\mathbb{E}\left[(X-c)^2\right]$ es $\mu_{\mathbf{x}}$.
\end{boxtheo}

\begin{boxcor}[Desigualdad de Markov I]
	Sean $\varphi$ una función monótona creciente y no negativa para los reales no negativos, $X$ una variable aleatoria, $a \ge 0$ y $\varphi(a) > 0$, entonces:
	\[
		\mathbb{P}\left( \left| X \right| \ge a \right) \le \frac{\mathbb{E}\left[ \varphi \left( \left| X \right|  \right)  \right] }{\varphi(a)}
	.\] 
\end{boxcor}

\begin{boxcor}[Desigualdad de Markov II]
	Sea $X$ una variable aleatoria y $a > 0$, entonces:
	\[
		\mathbb{P}\left( \left| X \right| \ge a  \right) \le \frac{\mathbb{E}\left[ \left| X \right|^r  \right] }{a^r}
	.\] 
\end{boxcor}

\begin{boxtheo}[Desigualdad de Chebyshev]
	Sean $X$ una variable aleatoria con $\mathbb{E}(X) = \mu$ y $\text{\normalfont Var}(X) = \sigma^2$, y $k > 0$, entonces:
	\[
		\mathbb{P}\left( \|X - \mu\|_{\alpha} \ge k\sigma \right) \le \frac{1}{k^2}
	,\] 
	o bien,
	\[
		\mathbb{P}\left( \|X - \mu\|_{\alpha} \ge k \right) \le \frac{\sigma^2}{k^2}
	,\] 
	donde $\| \cdot \|_{\alpha}$ es la norma $\alpha$.
\end{boxtheo}

\begin{boxrmk}[]
	Sean $X_1, X_2, \ldots, X_n$ una colección de $n\in\mathbb{N}$ variables aleatorias, podemos expresar las funciones de densidad conjunta en términos de densidades condicionales de la siguiente manera:
	\begin{align*}
		f(x_1, x_2, \ldots, x_n) &= f(x_1  \mid x_2, x_3, \ldots, x_n)\cdot  f(x_2, x_3, \ldots, x_n) \\
								 &=  f(x_1  \mid x_2, \ldots, x_n) \cdot f(x_2  \mid x_3, \ldots, x_n) \cdot \cdots \cdot f(x_{n-1}  \mid x_n) \cdot f(x_{n})
	.\end{align*}
\end{boxrmk}

\begin{boxtheo}[Ley de esperanzas iteradas]
	Sean $X$ y $Y$ dos variables aleatorias tales que $\mathbb{E}\left[ X \right] $ está definida y ambas están en el mismo espacio de probabilidad, entonces:
	\[
		\mathbb{E}\left[ X \right] = \mathbb{E}\left[ \mathbb{E}\left[ X  \mid Y \right]  \right] 
	.\] 
\end{boxtheo}

\begin{boxprop}[]
	Por la \textbf{ley de esperanzas iteradas}, tenemos:
	\[
		\text{\normalfont Var}(X) = \mathbb{E}\left[ \text{\normalfont Var}(X  \mid Y) \right] + \text{\normalfont Var} \left[ \mathbb{E}(Y  \mid X) \right] 
	.\] 
\end{boxprop}

\begin{boxprop}[]
	\[
		\mathbb{E}(XY) = \mathbb{E}\left[ X \mathbb{E}(Y | X) \right] 
	.\] 
\end{boxprop}



\newpage
\section{Econometría, modelos y datos}

\newpage
\section{Modelo de regresión lineal simple}

\newpage
\section{Análisis de los supuestos del modelo}

\newpage
\section{Modelo de regresión lineal múltiple}

\newpage
\section{Otros temas sobre el modelo de regresión lineal}

\vfill\eject
\columnbreak
\end{multicols}
\end{document}
