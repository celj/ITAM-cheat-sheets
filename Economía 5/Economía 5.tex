\documentclass[8pt,a4paper]{extarticle}
\usepackage[utf8]{inputenc}
\usepackage[spanish]{babel}
\usepackage[landscape, margin=1cm, bmargin=0.5cm, includefoot, footskip=0.5cm]{geometry}
\usepackage[textsize=tiny]{todonotes}
\usepackage{enumitem}
\usepackage{mdframed}
\usepackage{mathtools}
\usepackage{amsthm}
\usepackage{amssymb}
\usepackage{multicol,multirow}
\usepackage{subfiles}
\usepackage{tabularx}
\usepackage{bm}
\usepackage{xcolor}
\usepackage{graphicx}
\usepackage{accents}
\usepackage{pgfplots}
\usepackage{fancyhdr}
\usepackage[hidelinks]{hyperref}
\usepackage{nicefrac}
\usepackage{fontspec}
\usepackage{listings}

\newcommand{\cs}{Formulario}
\newcommand{\csof}{Formulario de }
\newcommand{\csAuthorName}{Carlos Lezama}
\newcommand{\csClass}{ }
\newcommand{\csClassCode}{ }
\newcommand{\csKeywords}{ }
\newcommand{\csTerm}{ }
\newcommand{\csSchool}{ITAM}

\pagestyle{fancy}
\renewcommand{\headrulewidth}{0pt}
\rhead{} 
\lhead{} 
\chead{} 
\cfoot{\csClass\ $\cdot$ \cs}
\lfoot{\csAuthorName}
\rfoot{Página \thepage}

\graphicspath{{./figures/}}

\setmonofont{JetBrainsMono Nerd Font Mono}[
    Contextuals = Alternate,
    Ligatures = TeX,
]

\lstset{
    basicstyle = \ttfamily,
    columns = flexible,
}

\usetikzlibrary{decorations.markings}
\pgfplotsset{compat=1.11}

\newmdtheoremenv [
	topline    = false,
	bottomline = false,
	leftline   = true,
	rightline  = false,
	linewidth  = 2.5pt,
	linecolor  = red!50
]{boxdef}{Definición}[section]

\mdtheorem [
	topline    = false,
	bottomline = false,
	leftline   = true,
	rightline  = false,
	linewidth  = 2.5pt,
	linecolor  = blue!40
]{boxtheo}{Teorema}[section]

\mdtheorem [
	topline    = false,
	bottomline = false,
	leftline   = true,
	rightline  = false,
	linewidth  = 2.5pt,
	linecolor  = black!20
]{boxprop}{Proposición}[section]

\mdtheorem [
	topline    = false,
	bottomline = false,
	leftline   = true,
	rightline  = false,
	linewidth  = 2.5pt,
	linecolor  = blue!40
]{boxlemma}{Lema}[section]

\mdtheorem [
	topline    = false,
	bottomline = false,
	leftline   = true,
	rightline  = false,
	linewidth  = 2.5pt,
	linecolor  = blue!40
]{boxcor}{Corolario}[section]

\mdtheorem [
	topline    = false,
	bottomline = false,
	leftline   = true,
	rightline  = false,
	linewidth  = 2.5pt,
	linecolor  = black!20
]{boxrmk}{Observación}[section]

\newlist{numberlist}{enumerate}{1}
\setlist[numberlist, 1]{label={\arabic*.}, itemsep=0em, leftmargin=*,labelindent=0.5em}

\newlist{eqlist}{enumerate}{1}
\setlist[eqlist, 1]{label={\normalfont (\roman*)},itemsep=-0.2em, leftmargin=*,labelindent=-0.5em}

\newlist{bulletlist}{itemize}{1}
\setlist[bulletlist, 1]{itemsep=0em, leftmargin=0.5em, label={·}}

\setlength{\parindent}{0em}

\newenvironment{Figure}
  {\par\medskip\noindent\minipage{\linewidth}}
  {\endminipage\par\medskip}

\newcommand\tab[1][0.5em]{\hspace*{#1}}

\newcommand{\sectionbreak}{\vfill\ \columnbreak}

\usepackage{array}
\newcolumntype{P}[1]{>{\centering\arraybackslash}p{#1}}
\newcolumntype{M}[1]{>{\centering\arraybackslash}m{#1}}


% Economics
\newcommand{\E}{\resizebox{0.2cm}{!}{$\varepsilon$}}
\newcommand{\EE}{\mathcal{E}}
\newcommand{\I}{\mathcal{I}}
\newcommand{\LL}{\mathcal{L}}
\newcommand{\F}{\mathcal{F}}
\newcommand{\MRS}{\text{\normalfont MRS}}

% Statistics

% Mathematics
\DeclareMathOperator*{\argmin}{\arg \min}
\DeclareMathOperator*{\argmax}{\arg \max}
\newcommand{\deq}{\stackrel{\text{\normalfont def}}{=}}
\newcommand{\ie}{\text{\normalfont i.e.}}
\newcommand{\sgn}{\text{\normalfont sgn}}



% Class info
\renewcommand{\csClass}{Economía 5}
\renewcommand{\csClassCode}{ECO - 22105}
\renewcommand{\csTerm}{Primavera 2021}
\renewcommand{\csKeywords}{ }

% PDF Metadata
\hypersetup{
    pdftitle={\csof \csClass},      
    pdfsubject={\csClass},      
    pdfauthor={\csAuthorName},  
    pdfkeywords={}              
}

% Begin document
\begin{document}

\begin{titlepage}
    \begin{center}
	\vspace*{1cm}
	\Huge
        \textbf{\csClass}
	\vspace{0.5cm} \\
	\Large
        \cs\ $\cdot$ \csTerm
        \vfill
        \csAuthorName
	\vspace{0.8cm}
        \csClassCode\\
        \csSchool     
    \end{center}
\end{titlepage}

\begin{multicols}{3}
\setcounter{page}{1}

\part{Producción y consumo}

\section{El modelo estático de producción y consumo}

\begin{boxdef}[Función de producción]
	La \textbf{función de producción} $f_j$ describe la relación entre la producción de bienes y la cantidad de trabajo requerido en la j-ésima empresa competitiva, y se denota:
	\[
		y_j = f_j(l) \text{ tal que } j \in J
	.\]
\end{boxdef}

\subsubsection*{Propiedades de la función de producción}

\begin{eqlist}
\item Creciente ($f'_j > 0$), i.e. el trabajo es siempre productivo.
\item Cóncava ($f''_j \le 0$), i.e. está sujeta a la ley de rendimientos marginales decrecientes.
\end{eqlist}

\subsection{El problema de la firma}

\begin{equation*}
	\max_{\{l\}}\ pf_j (l) - wl
\end{equation*}

\begin{center}
\begin{tabular}{ c l }
	\hline
	$f_j$ & Función de producción \\
	$l$	  & Nivel de empleo \\
	$p$   & Precio del bien final \\
	$w$   & Precio del trabajo (salario) \\
	\hline
\end{tabular}
\end{center}

\subsubsection*{Condición de optimalidad}

\[
	l : \qquad pf'_j(l_j(w,p)) = w
.\] 

\begin{boxdef}[Ganancias óptimas]
	Definimos las \textbf{ganancias óptimas} de la firma $j$ como sigue:
	\[
		\pi_j(w,p) = pf_j(l_j(w,p)) - wl_j(w,p)
	.\] 
\end{boxdef}

\begin{boxdef}[Demanda laboral]
	La solución $l_j$ de la condición de optimalidad del problema de la firma se conoce como \textbf{demanda laboral} de la firma $j$.
\end{boxdef}

\begin{boxdef}[Oferta de bienes]
	A la función $y_j(w,p)$ se le conoce como \textbf{oferta de bienes} de la empresa $j$.
\end{boxdef}

\begin{boxprop}
	Las funciones de \textbf{demanda laboral} y \textbf{oferta de bienes}  son homogéneas de grado 0.
\end{boxprop}

\begin{boxprop}
	La función de \textbf{ganancias óptimas} es homogénea de grado 1.
\end{boxprop}

\begin{boxdef}[Función de utilidad]
	Sea una función $u_i(h, c)$, esta representa la utilidad del i-ésimo  consumidor por ocio y consumo si, para cualquier par de alternativas $(h_0, c_0), (h_1, c_1) \in \R^2$, se tiene $u_i(h_0, c_0) < h_i(h_1, c_1)$ si y solo si  el consumidor en cuestión prefiere la canasta $(h_1, c_1)$ sobre la canasta $(h_0, c_0)$.
\end{boxdef}

\subsubsection*{Propiedades de la función de utilidad}

\begin{eqlist}
\item Continuamente diferenciable, i.e. existe $u'_i$ continua.
\item Creciente $(u'_i > 0)$.
\item Monótona.
\item Cuasicóncava.
\end{eqlist}

\subsection{El problema de los consumidores}

\begin{equation*}
\begin{aligned}
	\max_{\{h, c\}}\	  & u_i(h, c) \\
	\text{sujeto a} \quad & h + n = H_i, \\
						  & pc = wn + \sum_J \theta_{ij} \pi_j (w, p).
\end{aligned}
\end{equation*}

O bien,

\begin{equation*}
\begin{aligned}
	\max_{\{h, c\}}\	  & u_i(h, c) \\
	\text{sujeto a} \quad & wh + pc = wH_i + \sum_J \theta_{ij} \pi_j (w, p).
\end{aligned}
\end{equation*}

\begin{center}
\begin{tabular}{ c l }
	\hline
	$\theta_{ij}$								& Acciones de la firma $j$ \\
	$c$											& Consumo del bien final \\
	$\pi_j$										& Ganancias de la firma $j$ \\
	$wn$										& Ingreso laboral \\
	$\displaystyle \sum_J \theta_{ij} \pi_j (w,p)$			& Ingreso no laboral o de capital \\
	$p$											& Precio del bien final \\
	$w$											& Precio del trabajo (salario) \\
	$h + n = H_i$								& Restricción de tiempo \\
	$\displaystyle pc = wn + \sum_J \theta_{ij} \pi_j (w, p)$ & Restricción presupuestal \\
	$h$											& Tiempo dedicado al ocio \\
	$n$											& Tiempo dedicado al trabajo \\
	$H_i$										& Unidades de tiempo disponibles \\
	$wh + pc$									& Valor de mercado de la canasta de consumo \\
	\hline
\end{tabular}
\end{center}

\sectionbreak

\subsubsection*{Condiciones de optimalidad}

\begin{equation*}
\begin{aligned}
	h : \qquad & \frac{\partial u_i}{\partial h} (h^*, c^*) &=\ \lambda^* w,\\
	c : \qquad & \frac{\partial u_i}{\partial c} (h^*, c^*) &=\ \lambda^* p,\\
	\lambda : \qquad & wh^* + pc^* &=\ wH_i & + \sum_J \theta_{ij} \pi_j (w, p).
\end{aligned}
\end{equation*}

Si $h^*, c^* > 0$, en el óptimo: $\text{TMS} (h^*, c^*) = \displaystyle  \frac{w}{p}$ tal que 

\newpage

\part{Consumo en el tiempo}

\newpage

\part{Producción en el tiempo}

\newpage

\part{Economía abierta}

\newpage

\part{Inversión y capital}

\vfill\eject
\columnbreak
\end{multicols}
\end{document}
