% !TEX program = lualatex

\documentclass[8pt,a4paper]{extarticle}
\usepackage[utf8]{inputenc}
\usepackage[spanish]{babel}
\usepackage[landscape, margin=1cm, bmargin=0.5cm, includefoot, footskip=0.5cm]{geometry}
\usepackage[textsize=tiny]{todonotes}
\usepackage{enumitem}
\usepackage{mdframed}
\usepackage{mathtools}
\usepackage{amsthm}
\usepackage{amssymb}
\usepackage{multicol,multirow}
\usepackage{subfiles}
\usepackage{tabularx}
\usepackage{bm}
\usepackage{xcolor}
\usepackage{graphicx}
\usepackage{accents}
\usepackage{pgfplots}
\usepackage{fancyhdr}
\usepackage[hidelinks]{hyperref}
\usepackage{nicefrac}
\usepackage{fontspec}
\usepackage{listings}

\newcommand{\cs}{Formulario}
\newcommand{\csof}{Formulario de }
\newcommand{\csAuthorName}{Carlos Lezama}
\newcommand{\csClass}{ }
\newcommand{\csClassCode}{ }
\newcommand{\csKeywords}{ }
\newcommand{\csTerm}{ }
\newcommand{\csSchool}{ITAM}

\pagestyle{fancy}
\renewcommand{\headrulewidth}{0pt}
\rhead{} 
\lhead{} 
\chead{} 
\cfoot{\csClass\ $\cdot$ \cs}
\lfoot{\csAuthorName}
\rfoot{Página \thepage}

\graphicspath{{./figures/}}

\setmonofont{JetBrainsMono Nerd Font Mono}[
    Contextuals = Alternate,
    Ligatures = TeX,
]

\lstset{
    basicstyle = \ttfamily,
    columns = flexible,
}

\usetikzlibrary{decorations.markings}
\pgfplotsset{compat=1.11}

\newmdtheoremenv [
	topline    = false,
	bottomline = false,
	leftline   = true,
	rightline  = false,
	linewidth  = 2.5pt,
	linecolor  = red!50
]{boxdef}{Definición}[section]

\mdtheorem [
	topline    = false,
	bottomline = false,
	leftline   = true,
	rightline  = false,
	linewidth  = 2.5pt,
	linecolor  = blue!40
]{boxtheo}{Teorema}[section]

\mdtheorem [
	topline    = false,
	bottomline = false,
	leftline   = true,
	rightline  = false,
	linewidth  = 2.5pt,
	linecolor  = black!20
]{boxprop}{Proposición}[section]

\mdtheorem [
	topline    = false,
	bottomline = false,
	leftline   = true,
	rightline  = false,
	linewidth  = 2.5pt,
	linecolor  = blue!40
]{boxlemma}{Lema}[section]

\mdtheorem [
	topline    = false,
	bottomline = false,
	leftline   = true,
	rightline  = false,
	linewidth  = 2.5pt,
	linecolor  = blue!40
]{boxcor}{Corolario}[section]

\mdtheorem [
	topline    = false,
	bottomline = false,
	leftline   = true,
	rightline  = false,
	linewidth  = 2.5pt,
	linecolor  = black!20
]{boxrmk}{Observación}[section]

\newlist{numberlist}{enumerate}{1}
\setlist[numberlist, 1]{label={\arabic*.}, itemsep=0em, leftmargin=*,labelindent=0.5em}

\newlist{eqlist}{enumerate}{1}
\setlist[eqlist, 1]{label={\normalfont (\roman*)},itemsep=-0.2em, leftmargin=*,labelindent=-0.5em}

\newlist{bulletlist}{itemize}{1}
\setlist[bulletlist, 1]{itemsep=0em, leftmargin=0.5em, label={·}}

\setlength{\parindent}{0em}

\newenvironment{Figure}
  {\par\medskip\noindent\minipage{\linewidth}}
  {\endminipage\par\medskip}

\newcommand\tab[1][0.5em]{\hspace*{#1}}

\newcommand{\sectionbreak}{\vfill\ \columnbreak}

\usepackage{array}
\newcolumntype{P}[1]{>{\centering\arraybackslash}p{#1}}
\newcolumntype{M}[1]{>{\centering\arraybackslash}m{#1}}


% Economics
\newcommand{\E}{\resizebox{0.2cm}{!}{$\varepsilon$}}
\newcommand{\EE}{\mathcal{E}}
\newcommand{\I}{\mathcal{I}}
\newcommand{\LL}{\mathcal{L}}
\newcommand{\F}{\mathcal{F}}
\newcommand{\MRS}{\text{\normalfont MRS}}

% Statistics

% Mathematics
\DeclareMathOperator*{\argmin}{\arg \min}
\DeclareMathOperator*{\argmax}{\arg \max}
\newcommand{\deq}{\stackrel{\text{\normalfont def}}{=}}
\newcommand{\ie}{\text{\normalfont i.e.}}
\newcommand{\sgn}{\text{\normalfont sgn}}



% Class info
\renewcommand{\csClass}{Economía 5}
\renewcommand{\csClassCode}{ECO - 22105}
\renewcommand{\csTerm}{Primavera 2021}
\renewcommand{\csKeywords}{ }

% PDF Metadata
\hypersetup{
    pdftitle={\csof \csClass},      
    pdfsubject={\csClass},      
    pdfauthor={\csAuthorName},  
    pdfkeywords={}              
}

% Begin document
\begin{document}

\begin{titlepage}
    \begin{center}
	\vspace*{1cm}
	\Huge
        \textbf{\csClass}
	\vspace{0.5cm} \\
	\Large
        \cs\ $\cdot$ \csTerm
        \vfill
        \csAuthorName
	\vspace{0.8cm}
        \csClassCode\\
        \csSchool     
    \end{center}
\end{titlepage}

\begin{multicols}{3}
\setcounter{page}{1}

\part{Producción y consumo}

\section{El modelo estático de producción y consumo}

\begin{boxdef}[Función de producción]
	La \textbf{función de producción} $f_j$ describe la relación entre la producción de bienes y la cantidad de trabajo requerido en la j-ésima empresa competitiva, y se denota:
	\[
		y_j = f_j(l) \text{ tal que } j \in J
	.\]
\end{boxdef}

\subsubsection*{Propiedades de la función de producción}

\begin{eqlist}
\item Creciente, i.e. el trabajo es siempre productivo.
\item Cóncava, i.e. está sujeta a la ley de rendimientos marginales decrecientes.
\end{eqlist}

\subsection{El problema de la firma}

\begin{equation*}
	\max_{\{l\}} \quad pf_j (l) - wl
\end{equation*}

\begin{center}
\begin{tabular}{ c l }
	\hline
	$f_j$ & Función de producción \\
	$l$   & Nivel de empleo \\
	$p$   & Precio del bien final \\
	$w$   & Precio del trabajo (salario) \\
	\hline
\end{tabular}
\end{center}

\begin{boxdef}[Ganancias óptimas]
	Definimos las \textbf{ganancias óptimas} de la firma $j$ como sigue:
	\[
		\pi_j(w, p) = pf_j(l_j(w, p)) - wl_j(w, p)
	.\] 
\end{boxdef}

\begin{boxdef}[Demanda laboral]
	La solución $l_j$ de la condición de optimalidad del problema de la firma se conoce como \textbf{demanda laboral} de la firma $j$.
\end{boxdef}

\begin{boxdef}[Oferta de bienes]
	A la función $y_j(w, p)$ se le conoce como \textbf{oferta de bienes} de la empresa $j$.
\end{boxdef}

\begin{boxprop}
	Las funciones de \textbf{demanda laboral} y \textbf{oferta de bienes}  son homogéneas de grado 0.
\end{boxprop}

\begin{boxprop}
	La función de \textbf{ganancias óptimas} es homogénea de grado 1.
\end{boxprop}

\begin{boxdef}[Función de utilidad]
	Sea una función $u_i(h, c)$, esta representa la utilidad del i-ésimo  consumidor por ocio y consumo si, para cualquier par de alternativas $(h_0, c_0), (h_1, c_1) \in \mathbb{R}^2$, se tiene $u_i(h_0, c_0) < h_i(h_1, c_1)$ si y solo si  el consumidor en cuestión prefiere la canasta $(h_1, c_1)$ sobre la canasta $(h_0, c_0)$.
\end{boxdef}

\subsubsection*{Propiedades de la función de utilidad}

\begin{eqlist}
\item Continuamente diferenciable.
\item Creciente.
\item Monótona.
\item Cuasicóncava.
\end{eqlist}

\subsection{El problema de los consumidores}

\begin{equation*}
\begin{aligned}
	\max_{\{h, c\}} \quad & u_i(h, c) \\
	\text{sujeto a} \quad & \underbracket[0.5pt][0.5pt]{h + n = H_i}_{\text{\normalfont Restricción de tiempo}}, \\
						  & \underbracket[0.5pt]{pc = \overbracket[0.5pt]{w \; \cdot \;  n}^{\text{\normalfont Ingreso laboral}} + \overbracket[0.5pt]{\sum_{j \in J} \theta_{ij} \pi_j (w, p)}^{\text{\normalfont Ingreso no laboral}}.}_{\text{\normalfont Restricción presupuestaria}}
\end{aligned}
\end{equation*}

O bien,

\begin{equation*}
\begin{aligned}
	\max_{\{h, c\}} \quad & u_i(h, c) \\
	\text{sujeto a} \quad & \underbracket[0.5pt]{wh + pc}_{\substack{\text{\normalfont Valor de} \\ \text{\normalfont mercado de} \\ \text{\normalfont la canasta} \\\text{\normalfont de consumo} }} = \underbracket[0.5pt]{wH_i + \sum_{j \in J} \theta_{ij} \pi_j (w, p)}_{\text{\normalfont Riqueza}}.
\end{aligned}
\end{equation*}

\begin{center}
\begin{tabular}{ c l }
	\hline
	$\theta_{ij}$ & Acciones de la firma $j$ \\
	$c$           & Consumo del bien final \\
	$\pi_j$       & Ganancias de la firma $j$ \\
	$h$           & Tiempo dedicado al ocio \\
	$n$           & Tiempo dedicado al trabajo \\
	$H_i$         & Unidades de tiempo disponibles \\
	\hline
\end{tabular}
\end{center}

\sectionbreak

\begin{boxdef}[Demanda de ocio]
	La \textbf{demanda de ocio} es una de las soluciones al problema de los consumidores y se denota:
	\[
		h^* = h_i(w, p)
	.\] 
\end{boxdef}

\begin{boxdef}[Demanda de consumo]
	La \textbf{demanda de consumo} es una de las soluciones al problema de los consumidores y se denota:
	\[
		c^* = c_i(w, p)
	.\] 
\end{boxdef}

\begin{boxdef}[Oferta laboral]
	Dadas nuestras unidades de tiempo disponibles, $H_i$, y nuestra demanda de consumo $h_i(w, p)$, definimos la \textbf{oferta laboral} como sigue:
	\[
		n_i(w, p) = H_i - h_i(w, p)
	.\] 
\end{boxdef}

\subsection{Equilibrio competitivo}

\begin{boxdef}[Equilibrio competitivo]
	Definimos el \textbf{equilibrio competitivo} como un vector de precios $(w^*, p^*)$ y una asignación $\left( \left\{ l^*_j, y^*_j \right\}_{j \in J}, \left\{ h^*_i, c^*_i \right\}_{i \in I} \right)$ tales que:
	\begin{eqlist}
	\item Todas las cantidades son óptimas a los precios $(w^*, p^*)$.
		\begin{equation*}
			\begin{aligned}
				\ie \qquad & l^*_j = l_j(w^*, p^*), \\
				           & y^*_j = y_j(w^*, p^*), \\
						   & h^*_i = h_i(w^*, p^*), \\
					       & c^*_i = c_i(w^*, p^*).
			\end{aligned}
		\end{equation*}
	\item Las cantidades individuales vacían el mercado de bienes y el mercado laboral.
		\begin{equation*}
			\begin{aligned}
				\ie \qquad & \sum_{j \in J} y_i(w^*, p^*) = \sum_{i \in I} c_i(w^*, p^*),\ \text{y} \\
						   & \sum_{j \in J} l_j(w^*, p^*) = \sum_{i \in I} \left[ H_i - h_i(w^*, p^*) \right].
			\end{aligned}
		\end{equation*}
	\end{eqlist}
\end{boxdef}

\newpage

\subsection{Maximización del bienestar social o problema del \emph{planificador central}}

\begin{equation*}
	\begin{aligned}
		\max_{\{h, c, l\}} \quad   & u(h, c) \\
		\text{sujeto a} \quad      & h + l = H, \\
							       & c = f(l).
	\end{aligned}
\end{equation*}

O bien,

\[
	\max_{\{l\}} \quad u(H - l, f(l))
.\] 

\begin{boxdef}[Demanda de consumo agregada]
	Definimos la \textbf{demanda de consumo agregada} como sigue:
	\[
		C(w) = \sum_{i \in I} c_i(w)
	.\] 
\end{boxdef}

\begin{boxdef}[Demanda laboral agregada]
	Definimos la \textbf{demanda laboral agregada} como sigue:
	\[
		L(w) = \sum_{j \in J} l_j(w)
	.\] 
\end{boxdef}

\begin{boxdef}[Ganancias agregadas]
	Definimos las \textbf{ganancias agregadas} como sigue:
	\[
		\Pi(w) = \sum_{j \in J} \pi(w)
	.\] 
\end{boxdef}

\begin{boxdef}[Oferta laboral agregada]
	Definimos la \textbf{oferta laboral agregada} como sigue:
	\[
		N(w) = \sum_{i \in I} n_i(w)
	.\] 
\end{boxdef}

\begin{boxdef}[Producción agregada]
	Definimos la \textbf{producción agregada} como sigue:
	\[
		Y(w) = \sum_{j \in J} y_j(w)
	.\] 
\end{boxdef}

\emph{Nota:} en caso de encontrarnos con agentes heterogéneos, recordemos que cada subgrupo de consumidores, o empresas, con ciertas características obtendrá demandas, u ofertas, agregadas representativas tales que la demanda, u oferta, agregada de todos los agentes será la suma de los agregados representativos.

\sectionbreak

\section{Política fiscal en el modelo estático}

\begin{boxrmk}[]
	El gobierno financia el gasto público y las transferencias gubernamentales con la recaudación de impuestos.
	\[
		\ie \quad \underbracket[0.5pt]{ T = G + \Omega}_{\substack{\text{\normalfont Restricción} \\ \text{\normalfont presupuestaria} \\ \text{\normalfont del gobierno}}}
	.\] 
\end{boxrmk}

\subsection{Impuestos distorsivos}

\begin{boxprop}[]
	Sean $t_i$ y $\omega_i$ los impuestos y transferencias que el individuo $i$ paga y recibe, respectivamente. Para satisfacer la restricción presupuestaria del gobierno:
	\[
	\sum_{i \in I} \omega_i = \Omega = T = \sum_{i \in I} t_i
	.\] 
\end{boxprop}

\begin{boxprop}[]
	Cada individuo recibe una proporción fija $\varepsilon_i \ge 0$ de la recaudación.
	\[
	\ie \quad \omega_i = \varepsilon_i\Omega = \varepsilon_i T
	.\] 
\end{boxprop}

\subsubsection{Impuesto al ingreso}

El monto de los impuestos que el individuo $i$ paga es:

\[
	t_i = \tau_y \left( wn_i + \sum_{j \in J} \theta_{ij} \pi_j (w) \right) 
.\] 

\begin{boxdef}[Impuesto proporcional]
	Un \textbf{impuesto proporcional} o tasa impositiva única sobre la renta es un sistema de impuestos en el que el tipo de gravamen siempre será el mismo, independientemente del nivel de renta.
\end{boxdef}

\begin{boxdef}[Impuesto progresivo]
	Un \textbf{impuesto progresivo} es un sistema de impuestos en el cual se establece que a mayor nivel de renta, mayor será el porcentaje de impuestos a pagar sobre la base imponible. 
\end{boxdef}

\begin{boxprop}[]
	Cuando la tasa impositiva es única, la recaudación total de la economía es:
	\[
	T = \tau_y Y
	.\] 
\end{boxprop}

\begin{boxrmk}[]
	Dado que la incidencia legal de los impuestos recae en los hogares, el problema de las empresas no se modifica y los consumidores se enfrentan a una nueva restricción:
	\[
		c = (1 - \tau_y) \left( wn_i + \sum_{j \in J} \theta_{ij} \pi_j(w) \right) + \varepsilon_i\Omega
	.\] 
\end{boxrmk}

\subsubsection{Impuesto al consumo}

El monto de los impuestos que el individuo $i$ paga es:
\[
t_i = \tau_c c_i
.\] 

\begin{boxprop}[]
	La recaudación total de la economía es:
	\[
	T = \tau_c \sum_{i \in I} c_i
	.\] 
\end{boxprop}

\begin{boxrmk}[]
	Dado que la incidencia legal de los impuestos recae en los hogares, el problema de las empresas no se modifica y los consumidores se enfrentan a una nueva restricción:
	\[
		(1 + \tau_c)c_i = wn_i + \sum_{j \in J} \theta_{ij}\pi_j(w) + \varepsilon_i\Omega
	.\] 
\end{boxrmk}

\begin{boxprop}[]
	Si $\displaystyle 1 + \tau_c = \frac{1}{1 - \tau_y}$, $\tau_y < \tau_c$ y, por lo tanto, $T < T_c$.
\end{boxprop}

\begin{boxrmk}[]
	En esta simplificación, dado que la recaudación se reintegra a los consumidores vía transferencias \emph{lump sum}, esta no genera efecto riqueza alguno. El único efecto de los impuestos se da a través de la distorsión en el precio relativo de los bienes.
\end{boxrmk}

\newpage

\part{Consumo en el tiempo}

\newpage

\part{Producción en el tiempo}

\newpage

\part{Economía abierta}

\newpage

\part{Inversión y capital}

\vfill\eject
\columnbreak
\end{multicols}
\end{document}
