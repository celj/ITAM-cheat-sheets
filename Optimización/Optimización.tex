% !TEX program = lualatex

\documentclass[8pt,a4paper]{extarticle}
\usepackage[utf8]{inputenc}
\usepackage[spanish]{babel}
\usepackage[landscape, margin=1cm, bmargin=0.5cm, includefoot, footskip=0.5cm]{geometry}
\usepackage[textsize=tiny]{todonotes}
\usepackage{enumitem}
\usepackage{mdframed}
\usepackage{mathtools}
\usepackage{amsthm}
\usepackage{amssymb}
\usepackage{multicol,multirow}
\usepackage{subfiles}
\usepackage{tabularx}
\usepackage{bm}
\usepackage{xcolor}
\usepackage{graphicx}
\usepackage{accents}
\usepackage{pgfplots}
\usepackage{fancyhdr}
\usepackage[hidelinks]{hyperref}
\usepackage{nicefrac}

\newcommand{\cs}{Formulario}
\newcommand{\csof}{Formulario de }
\newcommand{\csAuthorName}{Carlos Lezama}
\newcommand{\csClass}{ }
\newcommand{\csClassCode}{ }
\newcommand{\csKeywords}{ }
\newcommand{\csTerm}{ }
\newcommand{\csSchool}{ITAM}

\pagestyle{fancy}
\renewcommand{\headrulewidth}{0pt}
\rhead{} 
\lhead{} 
\chead{} 
\cfoot{\csClass\ $\cdot$ \cs}
\lfoot{\csAuthorName}
\rfoot{Página \thepage}

\graphicspath{{./figures/}}

\usetikzlibrary{decorations.markings}
\pgfplotsset{compat=1.11}

\newmdtheoremenv [
	topline    = false,
	bottomline = false,
	leftline   = true,
	rightline  = false,
	linewidth  = 2.5pt,
	linecolor  = red!50
]{boxdef}{Definición}[section]

\mdtheorem [
	topline    = false,
	bottomline = false,
	leftline   = true,
	rightline  = false,
	linewidth  = 2.5pt,
	linecolor  = blue!40
]{boxtheo}{Teorema}[section]

\mdtheorem [
	topline    = false,
	bottomline = false,
	leftline   = true,
	rightline  = false,
	linewidth  = 2.5pt,
	linecolor  = black!20
]{boxprop}{Proposición}[section]

\mdtheorem [
	topline    = false,
	bottomline = false,
	leftline   = true,
	rightline  = false,
	linewidth  = 2.5pt,
	linecolor  = blue!40
]{boxlemma}{Lema}[section]

\mdtheorem [
	topline    = false,
	bottomline = false,
	leftline   = true,
	rightline  = false,
	linewidth  = 2.5pt,
	linecolor  = blue!40
]{boxcor}{Corolario}[section]

\newlist{numberlist}{enumerate}{1}
\setlist[numberlist, 1]{label={\arabic*.}, itemsep=0em, leftmargin=*,labelindent=0.5em}

\newlist{eqlist}{enumerate}{1}
\setlist[eqlist, 1]{label={\normalfont (\roman*)},itemsep=-0.2em, leftmargin=*,labelindent=-0.5em}

\newlist{bulletlist}{itemize}{1}
\setlist[bulletlist, 1]{itemsep=0em, leftmargin=0.5em, label={·}}

\setlength{\parindent}{0em}

\newenvironment{Figure}
  {\par\medskip\noindent\minipage{\linewidth}}
  {\endminipage\par\medskip}

\newcommand\tab[1][0.5em]{\hspace*{#1}}

\newcommand{\sectionbreak}{\vfill\ \columnbreak}

\usepackage{array}
\newcolumntype{P}[1]{>{\centering\arraybackslash}p{#1}}
\newcolumntype{M}[1]{>{\centering\arraybackslash}m{#1}}


% Economics
\newcommand{\E}{\resizebox{0.2cm}{!}{$\varepsilon$}}
\newcommand{\EE}{\mathcal{E}}
\newcommand{\I}{\mathcal{I}}
\newcommand{\LL}{\mathcal{L}}
\newcommand{\F}{\mathcal{F}}
\newcommand{\MRS}{\text{\normalfont MRS}}

% Statistics
\newcommand{\bias}{\text{\normalfont Bias}}
\newcommand{\corr}{\text{\normalfont Corr}}
\newcommand{\cov}{\text{\normalfont Cov}}
\newcommand{\var}{\text{\normalfont Var}}

% Mathematics
\newcommand{\ie}{\text{\normalfont i.e.}}
\newcommand{\sgn}{\text{\normalfont sgn}}



% Class info
\renewcommand{\csClass}{Optimización}
\renewcommand{\csClassCode}{MAT - 22211}
\renewcommand{\csTerm}{Primavera 2021}
\renewcommand{\csKeywords}{ }

% PDF Metadata
\hypersetup{
    pdftitle={\csof \csClass},      
    pdfsubject={\csClass},      
    pdfauthor={\csAuthorName},  
    pdfkeywords={}              
}

% Begin document
\begin{document}

\begin{titlepage}
	\begin{center}
		\vspace*{1cm}
		\Huge
		\textbf{\csClass}
		\vspace{0.5cm} \\
		\Large
		\cs\ $\cdot$ \csTerm
		\vfill
		\csAuthorName
		\vspace{0.8cm}
		\csClassCode\\
		\csSchool
	\end{center}
\end{titlepage}

\begin{multicols}{3}
	\setcounter{page}{1}

	\section{Optimización estática}

	\subsection{Análisis convexo}

	\begin{boxdef}[Conjunto convexo]
		Sea $X \subseteq \mathbb{R}^n$, decimos que $X$ es \textbf{convexo} si, para cualesquiera $\mathbf{x}, \mathbf{y} \in X$ y para toda $\lambda \in (0, 1)$, se cumple:
		\[
			\lambda \mathbf{x} + (1 - \lambda)\mathbf{y} \in X
			.\]
		Equivalentemente, decimos que $X$ es \textbf{convexo} si, para todas $a \in \partial X$ y $b \in X$, existe $\ell$ tal que $\langle b - a, \ell \rangle \le 0$; donde $\partial X$ es la \emph{frontera} de $X$ y $\langle \cdot\, , \cdot \rangle$ denota el producto punto.
	\end{boxdef}

	\begin{center}
		\begin{tikzpicture}
			\begin{scope}
				\draw[rotate=-45,fill=red!30] (0,0) ellipse (45pt and 30pt);
				\node [label={[yshift=-2.2cm]Conjunto convexo}] {};
			\end{scope}
			\begin{scope}[xshift=3.5cm]
				\useasboundingbox (-1,-1.35) rectangle (1.5,1.35);
				\draw[fill=red!30] (0,0) to [out=140,in=90] (-1,-1)
				to [out=-90,in=240] (0.8,-0.6)
				to [out=60,in=-60] (1.2,1.2)
				to [out=120,in=90] (0.3,0.7)
				to [out=-90,in=20] (0.3,0)
				to [out=200,in=-40] (0,0);
				\draw (-0.5,-0.5) -- (0.7,0.7);
				\fill (-0.5,-0.5) circle[radius=1.5pt];
				\fill (0.7,0.7) circle[radius=1.5pt];
				\node [label={[yshift=-2.2cm]Conjunto no convexo}] {};
			\end{scope}
		\end{tikzpicture}
	\end{center}

	\begin{boxprop}
		Sean $A$ y $B$ dos subconjuntos convexos de $\mathbb{R}^n$, entonces:
		\begin{eqlist}
			\item $A \cap B$ es convexo.
			\item $A + B = \{a + b : a \in A,\ b \in B\}$ es convexo.
			\item Para todo $k \in \mathbb{R}$, $kA = \{ka : a\in A\}$ es convexo.
		\end{eqlist}
	\end{boxprop}

	\begin{boxdef}[Función convexa]
		Sea $X \subseteq R^n$ un conjunto convexo, $f : X \to \mathbb{R}$ es una \textbf{función convexa} si, para toda $\mathbf{x}_1 \neq \mathbf{x}_2 \in X$ y toda $\lambda \in (0, 1)$, se tiene:
		\[
			f(\lambda \mathbf{x}_1 + (1 - \lambda)\mathbf{x}_2) \le \lambda f(\mathbf{x}_1) + (1 - \lambda) f(\mathbf{x}_2)
			.\]
		Si la desigualdad es estricta, se dice que la función es \textbf{estrictamente convexa}.
	\end{boxdef}

	\begin{boxdef}[Función cóncava]
		Sea $X \subseteq R^n$ un conjunto convexo, $f : X \to \mathbb{R}$ es una \textbf{función cóncava} si, para toda $\mathbf{x}_1 \neq \mathbf{x}_2 \in X$ y toda $\lambda \in (0, 1)$, se tiene:
		\[
			f(\lambda \mathbf{x}_1 + (1 - \lambda)\mathbf{x}_2) \ge \lambda f(\mathbf{x}_1) + (1 - \lambda) f(\mathbf{x}_2)
			.\]
		Si la desigualdad es estricta, se dice que la función es \textbf{estrictamente cóncava}.
	\end{boxdef}

	\begin{boxprop}
		Sean $X \subseteq \mathbb{R}^n$ un conjunto convexo, $f : X \to \mathbb{R}$ y $g : X \to \mathbb{R}$ dos funciones cóncavas, y $\alpha \in \mathbb{R}$, entonces:
		\begin{eqlist}
			\item $f$ es cóncava si $\alpha > 0$.
			\item $f$ es convexa si $\alpha < 0$.
			\item $f+g$ es cóncava.
		\end{eqlist}
	\end{boxprop}

	\begin{boxprop}
		Sean $X \subseteq \mathbb{R}^n$ un conjunto convexo, $g : X \to \mathbb{R}$ una función cóncava, y $h : Y \to \mathbb{R}$ una función cóncava y creciente tal que $g(X) \subseteq Y \subseteq \mathbb{R}$; entonces, $h \circ g$ es cóncava.
	\end{boxprop}

	\begin{boxdef}[Vector gradiente]
		Sea $f \in \mathcal{C}^1 (X)$, el \textbf{vector gradiente} de $f$ está dado por:
		\[
			\nabla f(\mathbf{x}) =
			\begin{pmatrix} \displaystyle \frac{\partial f}{\partial x_1} \\ \vdots \\ \displaystyle \frac{\partial f}{\partial x_n} \end{pmatrix}
			.\]
	\end{boxdef}

	\begin{boxdef}[Matriz hessiana]
		Sea $f \in \mathcal{C}^2 (X)$, se define la \textbf{matriz hessiana} de $f$ como $H_f(\mathbf{x})$, donde:
		\[
			H_f (\mathbf{x})_{i,j} = \frac{\partial^2 f(\mathbf{x})}{\partial x_i \partial x_j}
			.\]
	\end{boxdef}

	\begin{boxdef}[Serie de Taylor]
		\[
			T(\mathbf{x}) = \sum_{|\alpha| \ge 0} \frac{(\mathbf{x} - \mathbf{a})^\alpha}{\alpha\,!} \left(\partial^\alpha f \right)(\mathbf{a})
			.\]
	\end{boxdef}

	\begin{boxtheo}[de Taylor]
		Sea $f : \mathbb{R}^n \to \mathbb{R}$ tal que $f \in \mathcal{C}^k(\mathbf{a})$. Entonces, existe $h_{\alpha} : \mathbb{R}^n \to \mathbb{R}$ tal que:
		\[
			f(\mathbf{x}) = \sum_{|\alpha| \le k} \frac{\partial^\alpha f(\mathbf{a})}{\alpha\,!} (\mathbf{x} - \mathbf{a})^\alpha + \sum_{|\alpha| = k} h_{\alpha}(\mathbf{x})(\mathbf{x} - \mathbf{a})^\alpha
			,\]
		y
		\[
			\lim_{\mathbf{x} \to \mathbf{a}} h_{\alpha} (\mathbf{x}) = 0
			.\]
	\end{boxtheo}

	\begin{boxdef}[Matriz simétrica]
		Decimos que una matriz $A \in \mathcal{M}_{n \times n}$ es \textbf{simétrica} si y solo si:
		\[
			A = A^T
			.\]
	\end{boxdef}

	\begin{boxdef}[Matriz diagonalizable]
		Decimos que una matriz $A \in \mathcal{M}_{n \times n}$ es \textbf{diagonalizable} si y solo si existe una matriz $P  \in \mathcal{M}_{n \times n}$ invertible tal que $P^{-1}AP$ es diagonal.
	\end{boxdef}

	\begin{boxdef}[Matriz ortogonalmente diagonalizable]
		Decimos que una matriz $A \in \mathcal{M}_{n \times n}$ es \textbf{ortogonalmente diagonalizable} si y solo si existe una matriz $T  \in \mathcal{M}_{n \times n}$ invertible tal que $T^{-1}AT$ es diagonal y $T^{-1} = T^T$.
	\end{boxdef}

	\begin{boxtheo}[]
		Si $A \in \mathcal{M}_{n \times n}$ es simétrica, sus valores propios son reales.
	\end{boxtheo}

	\begin{boxtheo}[]
		Una matriz simétrica $A$ de tamaño $n \times n$ puede determinar la forma cuadrática $q_A$ de $n$ variables como sigue:
		\[
			q_A (\mathbf{x}) = \sum_{i = 1}^n \sum_{j = 1}^n a_{ij}x_ix_j = \mathbf{x}^T A \mathbf{x}
			.\]
	\end{boxtheo}

	\begin{boxcor}[Clasificación de formas cuadráticas]
		La matriz asociada a la forma cuadrática $q_A$ es:
		\begin{eqlist}
			\item definida positiva si $q_A(\mathbf{x}) > 0,\ \forall \mathbf{x} \neq 0$.
			\item definida negativa si $q_A(\mathbf{x}) < 0,\ \forall \mathbf{x} \neq 0$.
			\item semidefinida positiva si $q_A(\mathbf{x}) \ge 0,\ \forall \mathbf{x} \neq 0$.
			\item semidefinida negativa si $q_A(\mathbf{x}) \le 0,\ \forall \mathbf{x} \neq 0$.
			\item indefinida si $q_A(\mathbf{x})$ toma tanto valores positivos como negativos.
		\end{eqlist}
	\end{boxcor}

	\begin{boxdef}[Menores principales]
		Sea $A$ una matriz simétrica, los \textbf{menores principales} de esta matriz son los determinantes de todas las submatrices superiores izquierdas, es decir:
		\begin{align*}
			\left| A_1 \right| & = \left| (a_{11}) \right|,                            \\
			\left| A_2 \right| & = \left| \begin{pmatrix} a_{11} & a_{12} \\ a_{21} & a_{22} \end{pmatrix}  \right|,         \\
			\left| A_3 \right| & = \left| \begin{pmatrix} a_{11} & a_{12} & a_{13} \\ a_{21} & a_{22} & a_{23} \\ a_{31} & a_{32} & a_{33} \end{pmatrix}  \right|,\ \cdots
		\end{align*}
	\end{boxdef}

	\newpage

	\begin{boxtheo}[Criterio de menores principales]
		Sea $A$ una matriz simétrica asociada a la forma cuadrática $q_A(\mathbf{x})$, entonces:
		\begin{eqlist}
			\item\label{1}$q_A$ es definida positiva si y solo si los menores principales de $A$ son todos positivos.
			\item\label{2} $q_A$ es definida negativa si y solo si los menores principales de $A$ alternan signos de la forma:
			\[
				\left| A_1 \right| < 0, \left| A_2 \right| > 0, \left| A_3 \right| < 0, \ldots
				.\]
			\item $q_A$ es indefinida si $\left| A \right| \neq 0$, pero no se cumplen \ref{1} ni \ref{2}.
		\end{eqlist}
	\end{boxtheo}

	\begin{boxtheo}[Criterio de valores propios]
		Sea $A$ uns matriz simétrica asociada a la forma cuadrática $q_A(\mathbf{x})$, entonces:
		\begin{eqlist}
			\item $q_A$ es definida positiva si y solo si todos los valores propios de $A$ son positivos.
			\item $q_A$ es definida negativa si y solo si todos los valores propios de $A$ son negativos.
			\item $q_A$ es semidefinida positiva si y solo si todos los valores propios de $A$ son no negativos.
			\item $q_A$ es semidefinida negativa si y solo si todos los valores propios de $A$ son no positivos.
			\item $q_A$ es indefinida si y solo si la matriz $A$ tiene valores propios positivos y negativos.
		\end{eqlist}
	\end{boxtheo}

	\begin{boxtheo}[Criterio del primer orden]
		Sea $X \subseteq \mathbb{R}^n$ un conjunto convexo y $f : X \to \mathbb{R}$ tal que $f \in \mathcal{C}^1(X)$, entonces:
		\begin{eqlist}
			\item $f$ es convexa si y solo si, $\forall \mathbf{x}, \mathbf{y} \in X$, se tiene:
			\[
				f(\mathbf{y}) \ge f(\mathbf{x}) + \nabla f(\mathbf{x})(\mathbf{y} - \mathbf{x})
				.\]
			\item $f$ es cóncava si y solo si, $\forall \mathbf{x}, \mathbf{y} \in X$, se tiene:
			\[
				f(\mathbf{y}) \le f(\mathbf{x}) + \nabla f(\mathbf{x})(\mathbf{y} - \mathbf{x})
				.\]
		\end{eqlist}
	\end{boxtheo}

	\sectionbreak

	\begin{boxtheo}[Criterio de segundo orden]
		Sea $X \subseteq \mathbb{R}^n$ un conjunto convexo y $f : X \to \mathbb{R}$ tal que $f \in \mathcal{C}^2(X)$, entonces:
		\begin{eqlist}
			\item $f$ es convexa en $X$ si y solo si la forma cuadrática asociada a la matriz hessiana es semidefinida positiva.
			\item $f$ es estrictamente convexa en $X$ si y solo si la forma cuadrática asociada a la matriz hessiana es definida positiva.
			\item $f$ es cóncava en $X$ si y solo si la forma cuadrática asociada a la matriz hessiana es semidefinida negativa.
			\item $f$ es estrictamente cóncava en $X$ si y solo si la forma cuadrática asociada a la matriz hessiana es definida negativa.
		\end{eqlist}
	\end{boxtheo}

	\begin{boxdef}
		Sean $X \subseteq \mathbb{R}^n$ un conjunto convexo y $f : X \to \mathbb{R}$, definimos:
		\begin{bulletlist}
			\item la \textbf{gráfica} de $f$ como $$G_f = \{ (\mathbf{x}, r) \in X \times \mathbb{R} : f(\mathbf{x}) = r \} .$$
			\item el \textbf{epígrafo} de $f$ como $$E_f = \{ (\mathbf{x}, r) \in X \times \mathbb{R} : f(\mathbf{x}) \le r \} .$$
			\item el \textbf{hipógrafo} de $f$ como $$H_f = \{ (\mathbf{x}, r) \in X \times \mathbb{R} : f(\mathbf{x}) \ge r \} .$$
		\end{bulletlist}
	\end{boxdef}

	\begin{boxtheo}
		Sea $X \subseteq \mathbb{R}^n$ un conjunto convexo,
		\begin{eqlist}
			\item una función $f : X \to \mathbb{R}$ es convexa si y solo si $E_f$ es un conjunto convexo de $\mathbb{R}^{n+1}$.
			\item una función $f : X \to \mathbb{R}$ es cóncava si y solo si $H_f$ es un conjunto convexo de $\mathbb{R}^{n+1}$.
		\end{eqlist}
	\end{boxtheo}

	\begin{boxdef}
		Sean $X \subseteq \mathbb{R}^n$ un conjunto convexo y $f : X \to \mathbb{R}$, definimos:
		\begin{bulletlist}
			\item el \textbf{contorno} de $f$ en $k$ como $$C_f(k) = \{\mathbf{x} \in X : f(\mathbf{x}) = k\}.$$
			\item el \textbf{contorno superior} de $f$ en $k$ como $$CS_f(k) = \{\mathbf{x} \in X : f(\mathbf{x}) \ge k\}.$$
			\item el \textbf{contorno inferior} de $f$ en $k$ como $$CI_f(k) = \{\mathbf{x} \in X : f(\mathbf{x}) \le k\}.$$
		\end{bulletlist}
	\end{boxdef}

	\begin{boxtheo}[]
		\begin{eqlist}
			\item Si $f : X \to \mathbb{R}$ es cóncava en $A$, $CS_f(k)$ es convexo para toda $k$ en la imagen de $f$.
			\item Si $f : X \to \mathbb{R}$ es convexa en $A$, $CI_f(k)$ es convexo para toda $k$ en la imagen de $f$.
		\end{eqlist}
	\end{boxtheo}

	\begin{boxtheo}[]
		Sean $X \subseteq \mathbb{R}^n$ un conjunto convexo y $f : X \to \mathbb{R}$, entonces:
		\begin{eqlist}
			\item $f$ es cuasicóncava en $A$ si $CS_f (k)$ es convexo para toda $k$ en la imagen de $f$.
			\item $f$ es cuasiconvexa en $A$ si $CI_f (k)$ es convexo para toda $k$ en la imagen de $f$.
		\end{eqlist}
	\end{boxtheo}

	\begin{boxtheo}[]
		Sean $X \subseteq \mathbb{R}^n$ un conjunto convexo y $f : X \to \mathbb{R}$, entonces:
		\begin{eqlist}
			\item $f$ es cuasicóncava si $f$ es cóncava.
			\item $f$ es cuasiconvexa si $f$ es convexa.
		\end{eqlist}
	\end{boxtheo}

	\begin{boxtheo}[]
		Cualquier transformación monótona creciente de una función cuasiconvexa es cuasiconvexa. Asimismo, cualesquier transformación monótona creciente de una función cuasicóncava es cuasicóncava.
	\end{boxtheo}

	\newpage

	\section{Cálculo de variaciones}

	\newpage

	\section{Teoría de control óptimo}

	\newpage

	\section{Elementos de programación dinámica}

	\vfill\eject
	\columnbreak
\end{multicols}
\end{document}
