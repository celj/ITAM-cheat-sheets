\documentclass[8pt,a4paper]{extarticle}
\usepackage[utf8]{inputenc}
\usepackage[spanish]{babel}
\usepackage[landscape, margin=1cm, bmargin=0.5cm, includefoot, footskip=0.5cm]{geometry}
\usepackage[textsize=tiny]{todonotes}
\usepackage{enumitem}
\usepackage{mdframed}
\usepackage{mathtools}
\usepackage{amsthm}
\usepackage{amssymb}
\usepackage{multicol,multirow}
\usepackage{subfiles}
\usepackage{tabularx}
\usepackage{bm}
\usepackage{xcolor}
\usepackage{graphicx}
\usepackage{accents}
\usepackage{pgfplots}
\usepackage{fancyhdr}
\usepackage[hidelinks]{hyperref}
\usepackage{nicefrac}
\usepackage{fontspec}
\usepackage{listings}

\newcommand{\cs}{Formulario}
\newcommand{\csof}{Formulario de }
\newcommand{\csAuthorName}{Carlos Lezama}
\newcommand{\csClass}{ }
\newcommand{\csClassCode}{ }
\newcommand{\csKeywords}{ }
\newcommand{\csTerm}{ }
\newcommand{\csSchool}{ITAM}

\pagestyle{fancy}
\renewcommand{\headrulewidth}{0pt}
\rhead{} 
\lhead{} 
\chead{} 
\cfoot{\csClass\ $\cdot$ \cs}
\lfoot{\csAuthorName}
\rfoot{Página \thepage}

\graphicspath{{./figures/}}

\setmonofont{JetBrainsMono Nerd Font Mono}[
    Contextuals = Alternate,
    Ligatures = TeX,
]

\lstset{
    basicstyle = \ttfamily,
    columns = flexible,
}

\usetikzlibrary{decorations.markings}
\pgfplotsset{compat=1.11}

\newmdtheoremenv [
	topline    = false,
	bottomline = false,
	leftline   = true,
	rightline  = false,
	linewidth  = 2.5pt,
	linecolor  = red!50
]{boxdef}{Definición}[section]

\mdtheorem [
	topline    = false,
	bottomline = false,
	leftline   = true,
	rightline  = false,
	linewidth  = 2.5pt,
	linecolor  = blue!40
]{boxtheo}{Teorema}[section]

\mdtheorem [
	topline    = false,
	bottomline = false,
	leftline   = true,
	rightline  = false,
	linewidth  = 2.5pt,
	linecolor  = black!20
]{boxprop}{Proposición}[section]

\mdtheorem [
	topline    = false,
	bottomline = false,
	leftline   = true,
	rightline  = false,
	linewidth  = 2.5pt,
	linecolor  = blue!40
]{boxlemma}{Lema}[section]

\mdtheorem [
	topline    = false,
	bottomline = false,
	leftline   = true,
	rightline  = false,
	linewidth  = 2.5pt,
	linecolor  = blue!40
]{boxcor}{Corolario}[section]

\mdtheorem [
	topline    = false,
	bottomline = false,
	leftline   = true,
	rightline  = false,
	linewidth  = 2.5pt,
	linecolor  = black!20
]{boxrmk}{Observación}[section]

\newlist{numberlist}{enumerate}{1}
\setlist[numberlist, 1]{label={\arabic*.}, itemsep=0em, leftmargin=*,labelindent=0.5em}

\newlist{eqlist}{enumerate}{1}
\setlist[eqlist, 1]{label={\normalfont (\roman*)},itemsep=-0.2em, leftmargin=*,labelindent=-0.5em}

\newlist{bulletlist}{itemize}{1}
\setlist[bulletlist, 1]{itemsep=0em, leftmargin=0.5em, label={·}}

\setlength{\parindent}{0em}

\newenvironment{Figure}
  {\par\medskip\noindent\minipage{\linewidth}}
  {\endminipage\par\medskip}

\newcommand\tab[1][0.5em]{\hspace*{#1}}

\newcommand{\sectionbreak}{\vfill\ \columnbreak}

\usepackage{array}
\newcolumntype{P}[1]{>{\centering\arraybackslash}p{#1}}
\newcolumntype{M}[1]{>{\centering\arraybackslash}m{#1}}


% Economics
\newcommand{\E}{\resizebox{0.2cm}{!}{$\varepsilon$}}
\newcommand{\EE}{\mathcal{E}}
\newcommand{\I}{\mathcal{I}}
\newcommand{\LL}{\mathcal{L}}
\newcommand{\F}{\mathcal{F}}
\newcommand{\MRS}{\text{\normalfont MRS}}

% Statistics

% Mathematics
\DeclareMathOperator*{\argmin}{\arg \min}
\DeclareMathOperator*{\argmax}{\arg \max}
\newcommand{\deq}{\stackrel{\text{\normalfont def}}{=}}
\newcommand{\ie}{\text{\normalfont i.e.}}
\newcommand{\sgn}{\text{\normalfont sgn}}



% Class info
\renewcommand{\csClass}{Optimización}
\renewcommand{\csClassCode}{MAT - 22211}
\renewcommand{\csTerm}{Primavera 2021}
\renewcommand{\csKeywords}{ }

% PDF Metadata
\hypersetup{
    pdftitle={\csof \csClass},      
    pdfsubject={\csClass},      
    pdfauthor={\csAuthorName},  
    pdfkeywords={}              
}

% Begin document
\begin{document}

\begin{titlepage}
    \begin{center}
	\vspace*{1cm}
	\Huge
        \textbf{\csClass}
	\vspace{0.5cm} \\
	\Large
        \cs\ $\cdot$ \csTerm
        \vfill
        \csAuthorName
	\vspace{0.8cm}
        \csClassCode\\
        \csSchool     
    \end{center}
\end{titlepage}

\begin{multicols}{3}
\setcounter{page}{1}

\section{Optimización estática}

\subsection{Análisis convexo}

\begin{boxdef}[Conjunto convexo]
	Sea $X \subseteq \mathbb{R}^n$, decimos que $X$ es \textbf{convexo} si, para cualesquiera $x, y \in X$ y para toda $\lambda \in (0, 1)$, se cumple:
	\[
		\lambda x + (1 - \lambda)y \in X
	.\] 
	Equivalentemente, decimos que $X$ es \textbf{convexo} si, para todas $a \in \partial X$ y $b \in X$, existe $\ell$ tal que $\langle b - a, \ell \rangle \le 0$; donde $\partial X$ es la \emph{frontera} de $X$ y $\langle \cdot\, , \cdot \rangle$ denota el producto punto.
\end{boxdef}

\begin{center}
\begin{tikzpicture}
	\begin{scope}
		\draw[rotate=-45,fill=red!30] (0,0) ellipse (45pt and 30pt);
	\node [label={[yshift=-2.2cm]Conjunto convexo}] {};
	\end{scope}
	\begin{scope}[xshift=3.5cm]
		\useasboundingbox (-1,-1.35) rectangle (1.5,1.35);
		\draw[fill=red!30] (0,0) to [out=140,in=90] (-1,-1)
		to [out=-90,in=240] (0.8,-0.6)
		to [out=60,in=-60] (1.2,1.2)
		to [out=120,in=90] (0.3,0.7)
		to [out=-90,in=20] (0.3,0)
		to [out=200,in=-40] (0,0);
		\draw (-0.5,-0.5) -- (0.7,0.7);
		\fill (-0.5,-0.5) circle[radius=1.5pt];
		\fill (0.7,0.7) circle[radius=1.5pt];
		\node [label={[yshift=-2.2cm]Conjunto no convexo}] {};
	\end{scope}
\end{tikzpicture}
\end{center}

\begin{boxprop}
	Sean $A$ y $B$ dos subconjuntos convexos de $\mathbb{R}^n$, entonces:
	\begin{eqlist}
	\item $A \cap B$ es convexo.
	\item $A + B = \{a + b : a \in A,\ b \in B\}$ es convexo.
	\item Para todo $k \in \mathbb{R}$, $kA = \{ka : a\in A\}$ es convexo.
	\end{eqlist}
\end{boxprop}

\begin{boxdef}[Función convexa]
	Sea $X \subseteq R^n$ un conjunto convexo, $f : X \to \mathbb{R}$ es una \textbf{función convexa} si, para toda $x_1 \neq x_2 \in X$ y toda $\lambda \in (0, 1)$, se tiene:
	\[
		f(\lambda x_1 + (1 - \lambda)x_2) \le \lambda f(x_1) + (1 - \lambda) f(x_2)
	.\] 
	Si la desigualdad es estricta, se dice que la función es \textbf{estrictamente convexa}.
\end{boxdef}

\begin{boxdef}[Función cóncava]                                                                                         
    Sea $X \subseteq R^n$ un conjunto convexo, $f : X \to \mathbb{R}$ es una \textbf{función cóncava} si, para toda $x_1 \neq x_2 \in X$ y toda $\lambda \in (0, 1)$, se tiene:
    \[                                                                                                                  
        f(\lambda x_1 + (1 - \lambda)x_2) \ge \lambda f(x_1) + (1 - \lambda) f(x_2)                                     
    .\]                                                                                                                 
    Si la desigualdad es estricta, se dice que la función es \textbf{estrictamente cóncava}.                            
\end{boxdef}

\begin{boxdef}
	Sea $X \subseteq \mathbb{R}^n$ y $f : X \to \mathbb{R}$ una función, definimos:
	\begin{bulletlist}
	\item la \textbf{gráfica} de $f$ como $G_f = \{ (x, r) \in X \times \mathbb{R} : f(x) = r \} $.
	\item el \textbf{epígrafo} de $f$ como $E_f = \{ (x, r) \in X \times \mathbb{R} : f(x) \le r \}$.
	\item el \textbf{hipógrafo} de $f$ como $H_f = \{ (x, r) \in X \times \mathbb{R} : f(x) \ge r \}$.
	\end{bulletlist}
\end{boxdef}

\begin{boxtheo}
	Sea $X \subseteq \mathbb{R}^n$ un conjunto convexo,
	\begin{eqlist}
	\item una función $f : X \to \mathbb{R}$ es convexa si y solo si $E_f$ es un conjunto convexo de $\mathbb{R}^{n+1}$.
	\item una función $f : X \to \mathbb{R}$ es cóncava si y solo si $H_f$ es un conjunto convexo de $\mathbb{R}^{n+1}$.
	\end{eqlist}
\end{boxtheo}

\begin{boxprop}
	Sean $X \subseteq \mathbb{R}^n$ un conjunto convexo, $f : X \to \mathbb{R}$ y $g : X \to \mathbb{R}$ dos funciones cóncavas, y $\alpha \in \mathbb{R}$, entonces:
	\begin{eqlist}
	\item $f$ es cóncava si $\alpha > 0$.
	\item $f$ es convexa si $\alpha < 0$.
	\item $f+g$ es cóncava.
	\end{eqlist}
\end{boxprop}

\begin{boxprop}
	Sean $X \subseteq \mathbb{R}^n$ un conjunto convexo, $g : X \to \mathbb{R}$ una función cóncava, y $h : Y \to \mathbb{R}$ una función cóncava y creciente tal que $g(X) \subseteq Y \subseteq \mathbb{R}$; entonces, $h \circ g$ es cóncava.
\end{boxprop}

\newpage

\section{Cálculo de variaciones}

\newpage

\section{Teoría de control óptimo}

\newpage

\section{Elementos de programación dinámica}

\vfill\eject
\columnbreak
\end{multicols}
\end{document}
